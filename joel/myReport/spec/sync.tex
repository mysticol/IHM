%%This is a very basic article template.
%%There is just one section and two subsections.
\documentclass{article}

\usepackage[utf8]{inputenc}
\usepackage[francais]{babel}
\usepackage[T1]{fontenc}
\usepackage[nottoc, notlof, notlot]{tocbibind}
\usepackage[unicode=true,pdftex,colorlinks=true,linkcolor=black,urlcolor=black,citecolor=black]{hyperref}
\usepackage{natbib}
\usepackage{graphicx}


\title{Spécification Wikitty publication module de synchronisation}
\author{ Mano\"el \textsc{Fortun}}
\date{\today}



\begin{document}

\maketitle

\section{Définitions}

\subsection{Objets wikitty <-> Fichier }

Un fichier est défini par un nom, une extention, un contenu et un chemin.
dans wikitty publication les fichiers sont convertis en fonction de leurs type
en objet wikitty. les fichiers sources sont convertit en WikittyPubText et les
fichiers binaires(eg image, etc) en WikittyPubData. \\

Les deux types d'objet ont les mêmes attribut:
\begin{itemize}
\item Name: correspondant au nom du fichier
\item MimeType: crrespondant au type, qui donnera l'extension
\item Content: le contenu binaire pour pour les PubData et textuel pour les PubText
\end{itemize} 


Le mimetype donne l'extention par exemple pour un PubData si on a en mimetype ``image/png
`` alors l'extension du fichier associé sera ``.png''. Dans le cas d'un PubText
le mimetype donnera l'extension et le langage de la source par exemple si on a
en mimetype ``application/javascript'' le langage est javascript et l'extension
sera donc ``.js''. \\

A ces objets wikitty on associe un wikittyLabel, c'est un objet qui peut
contenir un ensemble de label différent, un label par exemple
``src.org.chorem.entities'' sert ici pour contenir le chemin menant au fichier
sur le file system. Un wikitty peut avoir un certain nombre de label, pour les
wikittyPub celà indiquera qu'ils appartiennent à plusieurs arborescence. 

\subsection{Propriétés}

\subsubsection{WikittyService cible}

L'adresse du wikitty service auprès duquel notre espace de travail(ensemble des
fichiers provenant d'un wikitty service) est lié sera stocké à la racine du dis
espace de travail(premier dossier), dans un dossier caché ``.wp\/''. Cette
adresse sera dans un fichier de propriété très certainement.

\subsubsection{Versions des wikitty et labels}

Tout objet wikitty dispose d'une version qui est modifier par le wikitty service
à chaque modification de l'objet wikitty, les wikittypub donc aussi. Cette
information de la version sera stocké dans un fichier de propriété dans le
dossier caché ``.wp\/'' afin que l'on garde trâce des versions des objets que
l'on aura transformé en fichier, celà pour mieux gérer les commit et update. \\

On conservera trace ainsi dans ce même fichier de propriété des labels, celà
permettra de pouvoir comparer les différences entre ce qui existe sur le file
system et les objets présents dans le wikitty service, et donc de voir les
suppressions, créations, etc.



\section{Fonctionnalités}



\subsection{Import}

La fonctionnalités import doit permettre d'importer auprès d'un wikitty service
un dossier, récurssivement ou non. Si on choisi récursivement alors tout les
sous dossier du dossier cible seront traité, sinon seulement les fichiers
contenu dans le dossier cible. Les fichiers sources seront transformé en
wikittyPubText et les binaires en WikittyPubData


\subsubsection{Définitions}

\begin{verbatim}
* ''import  [--recursion (true|false)] [url du WikittyService] [directory]
\end{verbatim}

\subsubsection{Prototype commande et exemples}

\subsection{CheckOut}
\subsubsection{Définitions}

->En tant qu'utilisateur de wikitty-publication je dois pouvoir 
checkout auprès d'un service wikitty en donnant un label,
je récupérais un fichier dont la localisation dépendra du label, sous
un nom dépendant de son name et une extension en fonction du mimetype.
Si l'opération checkout est demandé récursivement je récupérerais l'ensemble
des fichiers stocker sous le même label.

\subsubsection{Prototype commande et exemples}
\begin{verbatim}
* ''checkout [--recursion (true|false)] [url du WikittyService] [Label à
extraire] [directory local d'accueil]''
\end{verbatim}

\subsection{Relocate}
\subsubsection{Définitions}
->En tant qu'utilisateur de wikitty-publication je dois pouvoir changer l'adresse du service
wikitty cible de mon espace de travail. Cette opération ne pourra être éxécuté que à partir du
fichier racine de l'espace de travail, donc celui possédant le fichier caché contenant 
l'adresse du wikitty service. Celà implique que l'on enregistre dans le fichier caché de properties que l'on 
effectué cette action de relocate, et que du coup commit et update ne se passeront p'etre pas de la même façon.

\subsubsection{Prototype commande et exemples}

\begin{verbatim}
* ''relocate [nouvelle url du WikittyService par defaut] [directory a
relocaliser]''
\end{verbatim}

\subsection{Commit}
\subsubsection{Définitions}
->En tant qu'utilisateur de wikitty-publication je dois pouvoir 
commit auprès un service wikitty un ensemble cohérent(eg: une arborescence)
de fichier sous un même label, si un tel label existe déja sur le wiki on 
regardera la version, si la version du service est antérieur ou équivalente, alors
la précédente version sera écrasé par notre version.
Le commit commitera aussi les fichiers qui n'existait pas avant et fera les 
delete conséquent

\subsubsection{Prototype commande et exemples}

\begin{verbatim}
* ''commit [--recursion (true|false)] [--ws (url du WikittyService)] [répertoire
à pousser]''
\end{verbatim}



\subsection{Delete}
\subsubsection{Définitions}

->En tant qu'utilisateur de wikitty-publication je dois pouvoir delete 
un fichier ou dossier de mon espace de travail. Concrétement il faudra supprimer le label associé 
au wikitty, et si le wikitty se retrouve sans label, alors il devra être supprimé 
du wikitty service.

\subsubsection{Prototype commande et exemples}

\begin{verbatim}
* ''delete [--ws (url du WikittyService)] [répertoire ou fichier à supprimer]''
\end{verbatim}




\subsection{Update}
\subsubsection{Définitions}

->En tant qu'utilisateur de wikitty-publication je dois pouvoir update mon espace de travail,
pour les fichiers sources on effectue un diff pour mettre en relief les différences entre local 
et distant, pour les fichiers binaire on écrase, pour les fichiers qui localement ont été supprimé
on ne les remplacent pas. Pour les fichiers qui ont été supprimé sur le serveur, on les supprimes localement 
aussi.

\subsubsection{Prototype commande et exemples}

\begin{verbatim}
* ''update [--recursion (true|false)] [--ws (url du WikittyService)] [répertoire
à mettre à jour]''
\end{verbatim}







\end{document}
