% This text is proprietary.
% It's a part of presentation made by myself.
% It may not used commercial.
% The noncommercial use such as private and study is free
% Dec 2007
% Author: Sascha Frank 
% University Freiburg 
% www.informatik.uni-freiburg.de/~frank/
%
% 

\documentclass[12pt,a4paper,utf8x]{beamer}
\setbeamertemplate{navigation symbols}{}

\usepackage [frenchb]{babel}
\usepackage{color}
% Pour pouvoir utiliser 
\usepackage{ucs}
\usepackage[utf8x]{inputenc}

\usepackage{amsmath}

\usetheme{Marburg}

\addtobeamertemplate{footline}{\hfill\insertframenumber/\inserttotalframenumber} 

\beamersetuncovermixins{\opaqueness<1>{25}}{\opaqueness<2->{15}}
\begin{document}
\title{Wikitty Publication}  
\author{Manoël Fortun}

\begin{frame}
\titlepage
\end{frame}

\begin{frame}\tableofcontents	
\end{frame} 


\section{Wikitty} 
\subsection{C'est quoi ?}
\begin{frame}\frametitle{C'est quoi ?} 
\begin{itemize}
\item Base de données orientées document.
\item Clé $\to$ Valeur 
\item Des extensions
\item Entités générées grace à un modèle UML
\item champ nommé: WikittyPubText.name
\end{itemize}
\end{frame}

\subsection{Wikitty Service}
\begin{frame} \frametitle{Wikitty Service} 
\begin{itemize}
\item Restauration 
\item Sauvegarde
\item Recherche
\item Configuré par fichier de propriétés
\item "Complexité" de la base masqué
\end{itemize}
\end{frame}


\section{Scripting - JSR 223}
\subsection{Pardon ?}
\begin{frame}\frametitle{Pardon ?}
Concept très simple, la possibilité d'intrepréter un langage de script dans un
autre langage.\\

Par exemple, interpréter et éxécuter du \emph{javascript} dans du \emph{java}.
\end{frame}
\subsection{Bindings}
\begin{frame}\frametitle{Bindings}

Très important:
\begin{itemize}
\item Langage de script peut utiliser le langage qui l'interprète
\item Donc insérer du java dans du javascript
\item Méthode, Objets 
\item Map$\textless$String, Object$\textgreater$
\end{itemize}
\end{frame}


\section{Wikitty Publication} 
\subsection{Pourquoi faire ?}
\begin{frame}\frametitle{Pourquoi faire ?}
\begin{itemize}
\item Utilise le concept du scripting
\item Intégre un moteur d'évaluation de script
\item Script stocké sous forme de Wikitty
\item Wikitty Service en binding
\item Moteur de script en binding
\item WikittyPubText
\item WikittyPubData
\end{itemize}

\end{frame}

\subsection{Nouveau type de wikitty service}
\begin{frame}\frametitle{Nouveau type de wikitty service}
\begin{itemize}
\item wikitty service sur système de fichier
\item wikitty service sur jar
\item wikitty service fallback\pause
\end{itemize}\pause

\begin{figure}
\includegraphics[scale=0.5]{../image/multicontext.png} 
\caption{Service avec fallback}
\end{figure}
\end{frame}


\subsection{Nouvelles Fonctionnalités}
\subsubsection*{Synchronisation entre service}
\begin{frame}\frametitle{Synchronisation entre service}
\begin{itemize}
\item Transférer des wikitty d'un service à un autre
\item Basé sur les labels
\item Mise à jour
\item "Suppression"
\item "Déplacement" de wikitty
\end{itemize}
\pause
Exemple de label :
\verb!org.nuiton.wikitty!
\end{frame}

\begin{frame}
Exemple de synchronisation:
\begin{itemize}
\item cajo://localhost:1111\#com
\item cajo://wwikitty.nuiton.org:2222\#org
\end{itemize}

Les wikitty sous le label \verb!com! contenu sur le premier service vont
être envoyé sur le second service sous le label org.\\

Le label \verb!com.nuiton.wikitty! deviendra \verb!org.nuiton.wikitty! sur 
le second service.\\
\end{frame}

\subsubsection*{Externalisation}
\begin{frame}\frametitle{Externalisation}
\begin{itemize}
\item "Fixer" les wikitty
\item Compiler les scripts
\item Création d'un jar
\item WikittyPubTextCompiled
\end{itemize}
\end{frame}

\subsubsection*{Migration vers struts}
\begin{frame}\frametitle{Migration vers struts}
\begin{itemize}
\item Migration depuis une application en jsp "classic/simple"
\item Mécanisme de login/logout
\item action struts/interceptor
\item support struts pour les sessions etc
\end{itemize}
\end{frame}




\subsubsection*{Moteur d'évaluation}
\begin{frame}\frametitle{Moteur d'évaluation}

Moteur d'évaluation dans un navigateur à l'adresse:\\
\verb!/[contextData]/[contextApps]/eval/[mandatory]!

\begin{itemize}
\item mandatory pour retrouver le wikitty correspondant
\item eval, c'est le nom de l'action
\item contextApps, pour ne pas se tromper de wikitty.
\item contextData, pour trouver le bon wikitty service
\end{itemize}



\end{frame}

\begin{frame}\frametitle{Moteur d'évaluation}

Comment ça marche contextData:

\begin{figure}
\includegraphics[scale=0.5]{../image/propertiescontext.png} 
\caption{Surcharge des propriétés}
\end{figure}

\end{frame}

\begin{frame}\frametitle{Moteur d'évaluation}
Le mime type des WikittyPub détermine le traitement effectué.
\begin{itemize}
\item text/javascript, passera par l'évaluateur de javascript
\item image/png, renvoyé tel quel en mettant le bon type mime dans la réponse 
pour interprétation du navigateur
\item text/html.javascript, passera par un décorateur pour transformer le contenu
en text/javascript pour interprétation
\item text/java, sera compilé pour évaluation
\end{itemize}
\end{frame}

\begin{frame}\frametitle{Moteur d'évaluation}

Exemple:
\verb!codelutin/chorem/eval/Menu! 

On va "évaluer" le wikittyPub qui possède le nom "Menu" avec un label qui
commence par "chorem" et qui se trouve dans le service correspondant à Code Lutin.
Le rendu sera déterminé par le mime Type du wikittyPub correspondant.
\end{frame}





\section{Wikitty Struts} 
\begin{frame}\frametitle{Wikitty Struts}
Création d'une Tag lib struts pour une intégration facilité de wikitty dans des 
formulaire.
Deux possibilités d'utilisation
\begin{itemize}
\item Formulaire d'édition de wikitty, avec action pré-construite 
\item Intégration des champs de wikitty dans un formulaire, avec choix de la forme
de l'affichage en fonction du tag utilisé.
\end{itemize}

\end{frame}


\section{Plugin Maven}
\subsection{Un plugin?!}
\begin{frame}\frametitle{Un plugin?!}
\begin{itemize}
\item Motivation: Plus simple pour l'utilisateur
\item Clés en mains
\item Intègre de façon ciblée les fonctionnalités (externalize-synchronize)
\end{itemize}
%pom exemple
%liste des goals
\end{frame}	

\subsection{Les goals}
\begin{frame}\frametitle{Les goals}
\begin{itemize}
\item wp:init
\item wp:run
\item wp:deploy
\item wp:update
\item wp:jar
\item wp:jar-deploy
\end{itemize}

\end{frame}	


\end{document}
