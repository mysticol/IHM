\lstset{ %
language=XML,                % the language of the code
basicstyle=\footnotesize,       % the size of the fonts that are used for the code               % where to put the line-numbers
  % the size of the fonts that are used for the line-numbers
                  % the step between two line-numbers. If it's 1, each line 
keywordstyle=\color[rgb]{0,0,1},
                      % will be numbered
numbersep=5pt,                  % how far the line-numbers are from the code
showspaces=false,               % show spaces adding particular underscores
showstringspaces=false,         % underline spaces within strings
showtabs=false,                 % show tabs within strings adding particular underscores
tabsize=2,                      % sets default tabsize to 2 spaces
captionpos=b,                   % sets the caption-position to bottom
breaklines=true,                % sets automatic line breaking
breakatwhitespace=false,        % sets if automatic breaks should only happen at whitespace
title=\lstname,                 % show the filename of files included with \lstinputlisting;
                                % also try caption instead of title
escapeinside={\%*}{*)},         % if you want to add a comment with
morekeywords={project,modelVersion,groupId,description,build, plugin,plugins,configuration, applicationName, wikittyServiceUrl, artifactId,serverID,uploadUrl} 
}


\section{Wikitty Publication Maven Plugin}

Wikitty publication maven plugin, est un plugin maven intégrant les 
fonctionnalités développées pour wikitty publication, permettant une une
utilisation facilitées de ses fonctionnalités.

\subsection{Objectifs}

L'objectif de ce plugin était de rendre utilisable les outils dévellopé pour 
wikitty publication, comme la synchronisation ou l'externalisation, mais 
pour celà il fallait passer par les classes java associées. 

Or passer par les classes n'est pas "optimal" pour les développeurs, il faut
leur simplifier la vie, pour que ce genre de d'outil soit disponible 
immédiatement pour leurs besoins.

La meilleure façon pour celà est donc d'avoir un plugin maven associé à 
wikitty publication, pour que l'utilisation soit aisée, une solution clé en 
main.

L'utilisation la plus simple consiste donc en la création d'un pom élémentaire
avec les paramètres pour le plugin, et ensuite de dévelloper sont application
pour wikitty publication et d'utiliser les goals pour publier, essayer, etc.

Pom élémentaire :

\begin{lstlisting}
<?xml version="1.0" encoding="UTF-8"?>
<project xsi:schemaLocation="http://maven.apache.org/POM/4.0.0 http://maven.apache.org/xsd/maven-4.0.0.xsd" xmlns="http://maven.apache.org/POM/4.0.0"
    xmlns:xsi="http://www.w3.org/2001/XMLSchema-instance">
  <modelVersion>4.0.0</modelVersion>
  <groupId>org.exemple</groupId>
  <artifactId>stub</artifactId>
  <version>1.0</version>
  <description>exemple project</description>
  <build>
	<plugins>
	   <plugin>
		<groupId>org.nuiton.wikitty</groupId>
		<artifactId>wp-maven-plugin</artifactId>
		<version>3.2-SNAPSHOT</version>
		<configuration>
			<applicationName>example</applicationName>
			<wikittyServiceUrl>cajo://localhost:1111/wikitty</wikittyServiceUrl>
			<serverID>serveurTest</serveurID>
			<uploadUrl>http://localhost/</uploadUrl>
		</configuration>
	   </plugin>
	</plugins>
  </build>
</project>
\end{lstlisting}

%facilité la vie des dev pour le devellopement d'appli sur wikitty publication
%tout clés en mains 
%et tout et tout et tout.

\subsection{Goal}

Les goals suivant ont donc été fait pour ce plugin:

\begin{itemize}
\item wp:init
\item wp:run
\item wp:deploy
\item wp:update
\item wp:jar
\item wp:jar-deploy
\end{itemize}

La commande wp:init sert pour initialiser la bonne architecture pour le projet
d'application de wikitty publication. Comme suit :

\begin{verbatim}
src/main/wp
       #toute les pages
src/main/ressources/images
       #stocker les images
src/main/ressources/jar
       #stoker les binaires
\end{verbatim}

La commande wp:run pour lancer Wikitty Publication et y charger l'application
en cours de développement, cette commande lance un serveur avec wikitty publication
en utilisant un service type "fallback" qui est un service sur système de fichier,
l'application peut ainsi être modifiée/testée sans rédémarrer le serveur. 

La commande wp:deploy pour déployer l'application sur un wikitty service donné,
l'url de ce service est un paramètre dans le pom : wikittyServiceUrl. 
L'application sera déployé à l'aide du système de syncrhonisation, et le label
src/main tel que sur le file system sera remplacé par applicationName paramètre
aussi dans le pom, il représente le nom de l'application. Sur le wikitty service
on retrouvera cette organisation:
\begin{verbatim}
applicationName.wp
       #toute les pages
applicationName.ressources.images
       #stocker les images
applicationName.ressources.jar
       #stoker les binaires
\end{verbatim}


La commande wp:update sert pour mettre à jour le wikitty service sur lequel
on a précédement déployé l'application.

La commande wp:jar sert pour externaliser l'application dans un jar avec le
fonctionnement de l'externalisation, tout en substituant src/main par applicationName,
un jar parfaitement formé.

La commande wp:jar-deploy sert pour envoyer le jar sur un serveur ou copier
le jar ailleur sur le file system. Plus tard son role sera la publication 
des jar auprès d'un WikittyServiceJarLoader. Les paramètres de configuration dans
le pom serverID et uploadUrl sont respectivement pour retrouver dans les 
setting les éléments de configuration du serveur cible, et l'url ou l'on 
doit envoyer le jar.


\subsection{Fonctionnement}

L'écriture de ce plugin maven n'a pas été une tache trop compliqué, du fait de 
l'existence des implémentations des différents besoin, comme l'externalisation
ou la synchronisation, il aura fallu juste réutiliser ces fonctionnalités, voir
de corriger quelques signature pour permettre leur réutilisation.

Il aura aussi fallu réutiliser quelques plugin maven déjà existant comme jetty:run
pour le déploiment de l'application dans un serveur de test local. 

La création d'un plugin maven passe simplement par une implémentation java d'un
projet avec un pom définissant sa nature de plugin maven. Le reste c'est du java,
on implémente une interface "Mojo" défini dans maven, on doit aussi à l'aide de 
tag définir certain paramètre, d'autres tag servent aussi pour que le moteur
de maven puisse injecter des paramètres à l'éxécution du plugin permettant de
récupérer des composants/objet de maven comme le build ou le l'objet correspondant
au pom.

\lstset{ %
language=java,                % the language of the code
basicstyle=\footnotesize,       % the size of the fonts that are used for the code               % where to put the line-numbers
  % the size of the fonts that are used for the line-numbers
                  % the step between two line-numbers. If it's 1, each line 
keywordstyle=\color[rgb]{1,0,1},
                      % will be numbered
numbersep=5pt,                  % how far the line-numbers are from the code
showspaces=false,               % show spaces adding particular underscores
showstringspaces=false,         % underline spaces within strings
showtabs=false,                 % show tabs within strings adding particular underscores
tabsize=2,                      % sets default tabsize to 2 spaces
captionpos=b,                   % sets the caption-position to bottom
breaklines=true,                % sets automatic line breaking
breakatwhitespace=false,        % sets if automatic breaks should only happen at whitespace
title=\lstname,                 % show the filename of files included with \lstinputlisting;
                                % also try caption instead of title
escapeinside={\%*}{*)},         % if you want to add a comment with
morekeywords={@goal} 
}

Exemples:

\begin{lstlisting}
/**
 * @goal init
 */
public class WPInitMojo extends AbstractWPMojo {
...
}
\end{lstlisting}

Ici le tag goal permet de définir que la commande pour cette classe sera wp:init
(le wp le préfix du plugin en question). 

\begin{lstlisting}
    /**
     * The mandatory application name.
     *
     * @parameter expression="${wp.applicationName}"
     * @required
     * @since 1.2.5
     */
    protected String applicationName;

    /**
     * The component used for creating artifact instances.
     * @required
     * @component
     */
    private ArtifactFactory artifactFactory;

\end{lstlisting}

Le tag parameter associé à l'expression permet de récupérer le paramètre de 
configuration du plugin wp dans le pom(comme présenté précédement). 
Le tag component est un tag traité par le moteur de maven qui à l'éxécution
du plugin l'injection de l'objet demandé. C'est à ça que peuvent servir les 
tag dans un plugin maven. 


%rééutilisation de plug in existant

%Comment qu'on fait un plug in maven:
%héritage de mojo
%utilisation de taglet
%pour récupérer les éléments de l'éxécution 
%comme le pom et tout


