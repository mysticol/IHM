\section{Conclusion}

Ça fonctionne, ça fonctionnait aussi avant le stage, en soit le fait que ça 
fonctionne toujours n'est donc pas extraordinaire. Néanmoins maintenant des 
fonctionnalités supplémentaire sont venu se greffer autour du module pour le 
rendre utilisable réellement. Le prototype n'est plus seulement un prototype, 
et peut être plus facilement utilisable par la présence du plugin maven entre
autre.

Ainsi on peut maintenant assez facilement créer des applications dans Wikitty
Publication, c'est d'ailleurs prévu pour certaines application de chez Code
Lutin.

En ce qui concerne donc l'applicatif je pense avoir remplis mes objectifs, 
maintenant le projet est utilisable, pas parfait, mais libre, donc si son concept
intéresse, les gens peuvent l'améliorer pour leurs besoins et corriger ce qui ne
fonctionne pas. Des améliorations comme stocker dans le service les jar
entités sous forme de diagramme UML et à l'exécution les classes sont générées 
packagé en jar et ajoutées au classpath ou le support de d'autre langage.

Plus personnellement ce stage m'aura beaucoup apporté, techniquement j'aurais
beaucoup appris sur les bonnes pratiques, les mauvaises pratiques comme délégué
à son ide les mécanismes d'autres outil (inclure SVN dans eclipse ?! Plus jamais).

J'aurais appris à faire des points régulier sur l'avancement de mon travail,
tenir les délais que l'on me donnait, estimer la durée de certaines tâches, 
présenter mon travail au cours de réunion.

J'aurais pu appréhender plus facilement les problématiques de licence, et tout
ce qui concerne la propriété logicielle, et mieux comprendre les tenants et 
aboutissant du logiciel libre.
