\section{Annexes}


\subsection*{Extrait tld}

\lstset{ %
language=XML,                % the language of the code
basicstyle=\footnotesize,       % the size of the fonts that are used for the code               % where to put the line-numbers
  % the size of the fonts that are used for the line-numbers
                  % the step between two line-numbers. If it's 1, each line 
keywordstyle=\color[rgb]{0,0,1},
                      % will be numbered
numbersep=5pt,                  % how far the line-numbers are from the code
showspaces=false,               % show spaces adding particular underscores
showstringspaces=false,         % underline spaces within strings
showtabs=false,                 % show tabs within strings adding particular underscores
tabsize=2,                      % sets default tabsize to 2 spaces
captionpos=b,                   % sets the caption-position to bottom
breaklines=true,                % sets automatic line breaking
breakatwhitespace=false,        % sets if automatic breaks should only happen at whitespace
title=\lstname,                 % show the filename of files included with \lstinputlisting;
                                % also try caption instead of title
escapeinside={\%*}{*)},         % if you want to add a comment with
morekeywords={project,modelVersion,groupId,description,build, plugin,plugins,configuration, applicationName, wikittyServiceUrl, artifactId,serverID,uploadUrl} 
}



\begin{lstlisting}
       <tag>
        <name>textField</name>
        <description><![CDATA[tag to show wikitty field as textfield]]></description>        
        <tagclass>org.nuiton.wikitty.struts.tag.TextFieldTag</tagclass>
        <bodycontent>empty</bodycontent>
        <attribute>
            <description><![CDATA[id for the field]]></description>
            <name>id</name>
            <rtexprvalue>true</rtexprvalue>
        </attribute>
        <attribute>
            <description><![CDATA[name for html element. Ignored if tag used inside tag form.]]></description>
            <name>name</name>
            <rtexprvalue>true</rtexprvalue>
        </attribute>
        <attribute>
            <description><![CDATA[the wikitty to edit(needed if businessEntity not set)]]></description>
            <name>wikitty</name>
            <rtexprvalue>true</rtexprvalue>
            <required>false</required>
        </attribute>
        <attribute>
            <description><![CDATA[the businessEntity to edit(needed if wikitty not set)]]></description>
            <name>businessEntity</name>
            <rtexprvalue>true</rtexprvalue>
            <required>false</required>
        </attribute>
        <attribute>
            <description><![CDATA[the name of the field that have to be specialize]]></description>
            <name>fqFieldName</name>
            <rtexprvalue>true</rtexprvalue>
            <required>true</required>
        </attribute>
        <attribute>
            <description><![CDATA[the label of the field]]></description>
            <name>label</name>
            <rtexprvalue>true</rtexprvalue>
            <required>false</required>
        </attribute>
                <attribute>
            <description><![CDATA[if the textfiel have to hide the content, flag: true/false, default: false]]></description>
            <name>password</name>
            <rtexprvalue>true</rtexprvalue>
        </attribute>
    </tag>
\end{lstlisting}




\subsection*{Utilisation de la tagLib}


\subsection*{struts.xml}

Fichier de configuration xml de struts (partiel):

\lstset{ %
language=XML,                % the language of the code
basicstyle=\footnotesize,       % the size of the fonts that are used for the code               % where to put the line-numbers
  % the size of the fonts that are used for the line-numbers
                  % the step between two line-numbers. If it's 1, each line 
keywordstyle=\color[rgb]{0,0,1},
                      % will be numbered
numbersep=5pt,                  % how far the line-numbers are from the code
showspaces=false,               % show spaces adding particular underscores
showstringspaces=false,         % underline spaces within strings
showtabs=false,                 % show tabs within strings adding particular underscores
tabsize=2,                      % sets default tabsize to 2 spaces
captionpos=b,                   % sets the caption-position to bottom
breaklines=true,                % sets automatic line breaking
breakatwhitespace=false,        % sets if automatic breaks should only happen at whitespace
title=\lstname,                 % show the filename of files included with \lstinputlisting;
                                % also try caption instead of title
escapeinside={\%*}{*)},         % if you want to add a comment with
morekeywords={project,modelVersion,groupId,description,build, plugin,plugins,configuration, applicationName, wikittyServiceUrl, artifactId,serverID,uploadUrl} 
}



\begin{lstlisting}
<?xml version="1.0" encoding="UTF-8"?>
<!DOCTYPE struts PUBLIC 
	  "-//Apache Software Foundation//DTD Struts Configuration 2.0//EN"
	  "http://struts.apache.org/dtds/struts-2.0.dtd">
<struts>
    <constant name="struts.ognl.allowStaticMethodAccess" value="true" />
    <constant name="struts.enable.SlashesInActionNames" value="true" />

    <!-- Define a package for the restricted area must be logged to access -->
    <package name="restrictedArea" extends="publicArea">
        <interceptors>
            <interceptor name="login"
                class="org.nuiton.wikitty.publication.ui.interceptor.LoginInterceptor">
                <param name="error">/login_input.action</param>
            </interceptor>
            <interceptor-stack name="restrictedAreaStack">
                <interceptor-ref name="login" />
                <interceptor-ref name="publicAreaStack" />
            </interceptor-stack>
        </interceptors>
        <default-interceptor-ref name="restrictedAreaStack" />
    </package>
    <!-- Action aviable only to logged user extends="restrictedArea" -->
    <package name="publication" extends="publicArea">
        <action name="*/*/raw/*"
            class="org.nuiton.wikitty.publication.ui.action.PublicationActionRaw">
            <param name="contextData">{1}</param>
            <param name="contextApps">{2}</param>
            <param name="args">{3}</param>
            <result type="stream">
                <param name="contentType">${mimeType}</param>
                <param name="inputName">inputStream</param>
            </result>
        </action>

        <action name="*/*/eval/*"
            class="org.nuiton.wikitty.publication.ui.action.PublicationActionEval">
            <param name="contextData">{1}</param>
            <param name="contextApps">{2}</param>
            <param name="args">{3}</param>
            <result type="stream">
                <param name="contentType">${contentType}</param>
                <param name="inputName">inputStream</param>
            </result>
        </action>
    </package>
</struts>
\end{lstlisting}





\subsection*{ponay}
