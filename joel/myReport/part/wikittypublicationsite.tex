\section{Publication Site}

La partie site de Wikitty Publication concerne des fonctionnalités
destinées à l'administration et les éléments coeur du moteur de publication
comme l'interprétation du code contenu dans les WikittyPubText. Ces
fonctionnalités principales sont :
\begin{itemize}
\item Raw, qui permet l'affichage des WikittyPubData exemple des images
\item Edit, qui permet de créer ou d'éditer des Wikitty
\item Eval, qui permet l'évaluation du code contenu dans un WikittyPubText
\item View, qui permet l'affichage et la recherche des Wikitty
\end{itemize}

Un premier prototype avait été réalisé c'était la base de travail, ce prototype
était fait avec des jsp/servlet simplement sans support d'un quelconque
framework comme struts.

Le travail sur cette partie de publication se concentrait donc sur
l'amélioration de ce prototype sous différent aspect.


\subsection{Migration vers struts}

Le premier aspect de l'amélioration du prototype était de migrer les jsp/servlet
vers le framework struts. Struts est un donc un framework qui permet un bon
support du modèle MVC pour les applications web.

Struts permet au dévelloppeurs de se concentrer sur la vue et le modèle de son
application Web, à la charge de struts l'aspect contrôleur du modèle MVC. En
effet on peut définir l'enchainement des actions effectuées et la page qui sera
affiché en conséquence.

Dans struts on parle d'action, de package et d'intercepteur, les packages
regroupent les actions et les piles d'intercepteurs, 

Format des urls 
construction des url
action ?
Adapter les actions ?
Particulier de eval et raw ?


Principe de package dans struts


  /[context]/[action]/[mandatory_args]?[args key=value]
  
  eval
  raw
  edit
  view
  
  
    /eval/WikittyPubText.name:Wiki
  /eval/Wiki
  
  /raw/WikittyPubText.name:WikiLogo
  /raw/WikiLogo
  


Le travail à donc consister à supprimer les servlets, changer les jsp 



\subsection{Ajout d'un mécanisme de login/logout}

Le fonctionnement du mécanisme de droit est relativement simple, il est lié au
fonctionnement de struts. Avant d'accèder à une page d'un application sous
struts la requête passe par des intercepteurs, qui traite la requête en fonction
de ce qu'elle contient.

Dans le cas d'un système d'authentification avec struts on défini une pile
d'intercepteur personnalisée avec un intercepteur spécifique pour l'application
que l'on écrit en héritant d'une interface de de struts.




L'intercepteur vérifie donc que l'utilisateur est bien présent dans la session
courante, si non présent il redirige vers la page de Login, en conservant
l'adresse désiré par l'utilisateur de sorte que une redirection est effectuée
après un login correct.

Une fois ces 



gestion de droit
Login sur un wikitty contexte
avec intercepteur struts tout ça


\subsection{Amélioration des pages d'affichage et d'édition}

Les pages d'édition et d'affichage des wikitty étaient relativement simple, le
but étant une administration simple et efficace des wikitty. L'amélioration de
ces pages à consister en la correction des bugs présent déjà dans le prototype.

Par exemple le support pour les recherches dans wikitty dans la page view,
dans le prototype cette recherche était très limité et donc ne fonctionnait pas
très bien. De plus l'affichage des résultats nécessitait une amélioration.

Ensuite pour l'interface d'édition des wikitty j'ai intégré un décorateur de
Text Area qui permet une colorisation du contenu du champ, on peut choisir le
langage présent dans le champ pour avoir une colorisation, de l'indentation, des
outils de recherche/remplacer et d'autre, basiquement une sorte d'ide pour
faciliter le dévellopement de code dans les WikittyPubText.




page d'édition gestion des champs collection de string
pas grand chose de plus ici en fait.

Intégration d'un décorateur de text area pour code mirror ui, 

 

Page de view avec la recherche et d'autre truc
comme des liens qui manquait, plus d'info à afficher ou des trucs comme ça.




\subsection{Mécanisme de multicontext}

Cette fonctionnalité consiste en fait à l'encapsulation par un wikitty service
de deux autres wikitty service, quelque soit leurs natures.

Ce multi contexte éxécute ses recherches un wikitty service principal, et
compléte ses recherches si besoin avec les données du wikitty service dit
fallback.

De même l'écriture s'effectue sur le service principale, la restoration d'un
wikitty sur le principal et si il n'est pas dessus on va interroger le fallback
service.

%coller un schéma ici ?  ouai coller un schéma

Son utilisation n'est pas nécessairement limité à Wikitty Publication, il s'agit
finalement d'une implémentation différente de l'interface wikitty service. 

Ce mécanisme peut permettre la surcharge d'un wikitty, dans le sens ou on peut
modifier un wikitty qui aura été chargé depuis le fallback, mais à la sauvegarde
il sera restaurer depuis le wikitty principal, puisque celui ci est prioritaire.

On peut avoir ainsi un wikitty service statique, et un autre dynamique, par
exemple le wikitty service statique qui serait partagé et utilisé par d'autre
wikitty service multicontext, une base commune de wikitty.

