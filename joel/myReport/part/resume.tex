\section*{Résumé}

Ce rapport présente le stage que j'ai effectué dans le cadre de mon master 2
ALMA, stage effectué chez Code Lutin, une société de service en logiciel libre.
Mon travail a été articulé autour de Wikitty, une librairie libre 
développée à l'origine par l'entreprise. Il s'agit d'un système de base de 
données orienté colonnes, soit clé$\to$valeur. J'ai travaillé sur un module de 
ce projet : Wikitty Publication. Le but de ce module est de stocker des applications
dans le système de stockage Wikitty, tout comme les données. En plus du stockage
l'application doit pouvoir être exécutée et utilisée. Le but étant d'avoir une 
application qui peut se modifier comme elle s'exécute simplement à travers un 
navigateur comme pour un wiki. Au départ du stage il y avait un prototype simple,
fonctionnel. L'objectif du stage a été de le transformer en un module fonctionnel,
avec fonctionnalités un peu plus avancées. Ce concept d'application 
se base sur le scripting et la possibilité d'interpréter des langages de script
avec d'autre langage, dans notre cas ici pour le prototype du javascript
interprété et exécuté par du java, le tout formant une application web complète.\\

Mots clés: Libre, Scripting, Wiki, Wikitty, JAVA




