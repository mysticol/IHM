\section{Wikitty Publication Externalize}

La partie externalize de wikitty publication concerne les fonctionnalités 
d'externalisation des wikitty. C'est à dire prendre des wikitty stocker sur un 
système de fichier, via le service correspondant précédement dévellopé, compiler
ces wikitty et les stockers dans un jar.

L'objectif recherché par cette partie de wikitty publication est la performance,
en ayant du code compilé on gagne évidement en rapidité d'éxécution des wikitty
compilé. Et du code non modifiable, par exemple dans le cadre de l'utilisation 
multicontexte, on pourrait avoir un service s'appuyant sur des wikitty stocké 
dans un jar.

Comme pour la synchronisation vers file system, sont concernés par ce méchanisme
uniquement les Wikitty représentable sur file system, comme les wikittyPubData,
qui ne seront pas affecté par le méchanisme, et les wikittyPubText qui eux seront
compilé dans le processus d'externalisation.

Un des objectifs, plus secondaire, est la possibilité d'intégrer ce méchanisme
en tache Ant, ou un plug in maven.

\subsection{Une nouvelle extension de wikittyPub}

Les wikittyPubText qui auront été externalizé, et donc compilé, doivent toujours 
être considéré comme des wikitty, avec toutes les propriétés liées. Pour ce 
faire on défini une extension de wikittyPubText qui contiendra un champ nommé 
ByteCode de type tableau de byte.

Ainsi pour ce type de wikitty d'un coté on aura le code original du 
wikittyPubText dans le champ content et le code java compilé dans le champ
ByteCode. Celà permet d'avoir l'original dans le cas où l'on voudrait écrire
un nouveau wikitty avec le même code, ou dans le cas d'une surcharge si 
multicontexte.

\subsection{Mechanisme de transformation/compilation}

Tout le méchanisme de transformation des wikittyPubText en wikittyPubTextCompiled
doit doit respecter la finalité de ces wikitty, c'est à dire que le code qu'ils
contiennent soit exécutable et exécuté. 

Quelque soit le contenu des wikittyPubText ils sont compilé, normalement pour les
langages visé, tel le javascript on ne peut pas le compilé comme on le pourrais 
pour du java. Ce qui est fait c'est que ce code est encapsuler dans un morceau
de java qui invoque le script engine chargé de son exécution. 

Pour les langages comme le java on met simplement le code dans une méthode qui 
sera appellée quand on évaluera le code.

Pour ces deux cas le code java qui décore est presque le même, on à une classe
abstraite de décoration qui possède une méthode abstraitre eval, qui prend en 
paramètre la map contenant les Bindings[ref binding] nécessaire à l'éxécution 
du code.

% code décorer java
% code décoré non java TODO

En dehors de la compilation et décoration des wikitty, le mechanisme est 
relativement simple. On sélectionne tous les wikitty du dossier local où on
se trouve avec le service file system, on créer un dossier temporaire, 
on y écrit les wikitty, on compile ceux qui doivent l'être, on y met le fichier
wikitty original avec le code qu'il contient, un .java et le .class, puis le 
dossier est packager en jar, et le dossier temporaire supprimé.


\subsection{Sauvegarde des propriétés des wikitty}

De la même façon que pour le stockage des wikitty sur file system afin de 
pouvoir réutiliser les wikitty et garantir que ce sont toujours des wikitty,
il faut une inversabilité de la sauvegarde, et donc conserver des propriétés des
wikitty.

Contrairement à la sauvegarde des wikitty sur File System, ici on peut se 
permettre de centraliser toutes les informations dans les mêmes fichiers de 
propriété. 

Les propriétés conservée seront: les ids, les versions ainsi que les extensions
des fichiers correspondant. On utilise deux fichiers de propriétés distinct, 
un pour les id et les localisations des wikitty, et un autre pour les autres 
propriété.

Exemple: % plutot mettre une image

jar:
\begin{verbatim}
+jar
|id.properties
|meta.properties
|+wp
||Wiki.wp
||Wiki.java
||Wiki.class
||Logo.png
||Stat.wp
||Stat.java
||Stat.class
||Home.wp
||Home.java
||Home.class
||WikiMenu.wp
||WikiMenu.java
||WikiMenu.class
||+dossier
|||dummy.wp
|||dummy.java
|||dummy.class
|||uneAutreImage.jpg
|||uneImage.jpg
\end{verbatim}

Fichiers de propriétés correspondant:


ids.properties:
\begin{verbatim}
62c7d0cd-91c2-4607-bdcf-f0cac6c78d6e=wp/Wiki
4120d1ba-18ac-4ac5-a329-7f515f704fd0=wp/Logo
8225d3cb-beb3-4a85-a4ea-4b0e660fab96=wp/Stat
1968c510-6ba6-4de7-bbb0-67e029119520=wp/WikiMenu
2ca85aa8-2869-4ad6-8976-c0b0b5526934=wp/dossier/dummy
12f55beb-af63-49b1-9d74-62131bd0a67a=wp/Home
b7599025-f425-4dcf-88e6-55e154430d7e=wp/dossier/uneAutreImage
e443dbbe-b461-41bd-b5b6-b612e964cb0d=wp/dossier/uneImage
\end{verbatim}


meta.properties
\begin{verbatim}
#
#Tue Jul 12 11:19:46 CEST 2011
path.separator=/
8225d3cb-beb3-4a85-a4ea-4b0e660fab96.extension=wp
4120d1ba-18ac-4ac5-a329-7f515f704fd0.extension=png
12f55beb-af63-49b1-9d74-62131bd0a67a.extension=wp
1968c510-6ba6-4de7-bbb0-67e029119520.extension=wp
62c7d0cd-91c2-4607-bdcf-f0cac6c78d6e.extension=wp
e443dbbe-b461-41bd-b5b6-b612e964cb0d.extension=jpg
b7599025-f425-4dcf-88e6-55e154430d7e.extension=jpg
2ca85aa8-2869-4ad6-8976-c0b0b5526934.extension=wp
12f55beb-af63-49b1-9d74-62131bd0a67a.version=1.0
1968c510-6ba6-4de7-bbb0-67e029119520.version=1.0
2ca85aa8-2869-4ad6-8976-c0b0b5526934.version=1.0
4120d1ba-18ac-4ac5-a329-7f515f704fd0.version=1.0
62c7d0cd-91c2-4607-bdcf-f0cac6c78d6e.version=1.0
b7599025-f425-4dcf-88e6-55e154430d7e.version=1.0
e443dbbe-b461-41bd-b5b6-b612e964cb0d.version=1.0
8225d3cb-beb3-4a85-a4ea-4b0e660fab96.version=1.0
\end{verbatim}


Sous l'id on sauvegarde le chemin pour retrouver le wikitty correspondant, 
avec l'extension on peut retrouver le mimeType correspondant et le fichier
initial contenant le code, puisque naturellement on aura au même endroit
le fichier initial, le ".java" correspondant et le ".class". 

\subsection{Wikitty service sur jar}

Une fois que les wikitty sont sauvegardé dans un jar, on doit pouvoir les 
récupérer pour les utiliser comme n'importe quels autres wikitty. Pour celà il 
faut donc un wikitty service qui s'appuiera sur un jar et qui exposera les 
wikitty de la même façon.

Ce service propose donc les méthodes de recherche, et de restauration de wikitty,
mais évidement pas les méthodes de sauvegarde, puisque le but est bien d'avoir
des wikitty non modifiable.

La création de ce service à été facilité par l'existence du service sur file
system, puisque le fonctionnement est relativement le même, pour la recherche
on ne délègue pas à une base de donnée, on manipule directement les objets en
java, donc pour la recherche c'est le même méchanisme que pour le service sur
File system.



