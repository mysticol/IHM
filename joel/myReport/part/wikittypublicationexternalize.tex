\subsection{Externalize}


Le but d'externaliser les wikitty est de précompiler les wikitty pour gagner en 
performance et fixer les wikittyPubText empêchant leurs modifications. 
Ainsi avoir un wikitty service contenant le code applicatif "fixé" et des 
wikitty service qui viennent se connecter dessus et exploiter l'application avec
les wikitty "données" que eux possèdent.

La précompilation devrait permettre plus facilement un tel fonctionnement. 

Ce méchanisme sous entends qu'en plus de les précompiler on stocke les wikitty
compilé dans un jar, pour empêcher la modification.


Lecture/écriture
++++++++++++++++

Ecriture
--------

L'écriture du jar et des wikitty compilé se fera par l'intermédiaire d'un
main java, qui pourra ensuite être étendu en plug in maven ou tâche ant.
Son role sera d'a partir d'un repos filesystem de wikitty, donc selon 
le modèle de sauvegarde de wikittyPublicationFileSystem on compile les wikitty
et on les sauvegardes dans un jar en y joignant les sources des wikitty compilés.

Lecture
-------

La lecture de ce jar avec les wikitty sera à la charge d'un wikitty service qui
implémentera donc les méthodes de recherche et restauration de wikitty dans le
jar, il n'aura pas besoin des methodes de sauvegarde puisque le jar n'est 
pas sensé être modifié.


Un nouveau type de wikittyPub
+++++++++++++++++++++++++++++

Pour que les wikitty compilés soient utilisés et éxécuté, il faut qu'ils soient
encore des wikitty, ils ne peuvent plus être considéré comme des wikittyPubText,
et on ne doit pas pouvoir les modifier.

Il faut une nouvelle extension de wikittyPubText wikittyPubTextCompiled.

Qui aurait un champ qui contiendrait le byteCode java.

Cette extension serait utilisé en plus de l'extension wikittyPubText, évidement
puisque c'est une extension, mais celà signifie que dans le jar on aura les 
source du wikitty, et donc à la restoration on pourra remplir le champ content 
du wikittyPubText.



Conservation des propriétés des wikitty
+++++++++++++++++++++++++++++++++++++++

Comme pour le stockage des wikitty sur file system dans wikitty publication, 
afin de pouvoir réutiliser les wikitty il faut pouvoir conserver certaines 
propriété des wikitty, comme l'id et la version.

Dans le jar on sauvegardera donc les fichiers correspondant aux classes 
compilés et les sources des wikitty ayant servit à la compilation, donc des
wikittyPubText. 

Une solution serait comme pour les wikitty dans le file système de sauvegarde
les propriétés dans des fichiers de propriétés. Mais dans le minimum possible,
on aurait à la racine du jar dans un sous dossier externalIndex deux fichiers de 
propriétés pour les attributs des wikitty.

Le fait que tout soit enregistré dans seulement deux fichiers de propriétés 
permet de ne pas avoir à chercher dans toute l'arborescence du jar pour 
retrouver les wikitty, contrairement au cas de la synchronisation où c'était 
nécessaire.

Exemple:
--------


jar:

+jar
|externalIndex/
||id.properties
||meta.properties
|+label
||unwikittyCompilé.class
||unautre.class    
||+chausette
|||autre.class
|||fff.class
|wi.jpg

Fichiers de propriétés correspondant:
-------------------------------------

id.properties:
11daz5facz=/label/unwikittyCompilé # je ne sais pas si on stocke le chemin
jbdub1dza8=/label/unautre         # direct vers le class ou le source
dadzadzac=/label/chausette/autre
dcdzcve678=/label/chausette/fff
oniifdefe4=/wi.jpg


meta.properties
11daz5facz.version=8
jbdub1dza8.version=2
dadzadzac.version=3
dcdzcve678.version=1
oniifdefe4.version=1.2
11daz5facz.extension=wp
jbdub1dza8.extension=png
dadzadzac.extension=js
dcdzcve678.extension=wp
oniifdefe4.extension=jpg

On sauvegarde l'extension qui permet de récupérer le fichier original avec le
chemin enregistré dans ids.properties

Mechanisme de transformation
++++++++++++++++++++++++++++

Le mechanisme de transformation d'un wikittyPubText en wikitty externalizé 
implique de décorer le code du wikittyPubText par du code java pour le compiler 
et le stocker sous forme de classe.

Le mechanisme de décoration/compilation s'appliquera pour tout les langages, 
pour ceux qui ne sont pas compilable, il sera chargé de décorer le script
avec les éléments (scriptEngine, etc) nécessaire à son interprétation,
et ce code proposera une méthode qui en entré à besoin des éléments de context
d'interprétation et renverra le résultat de l'exécution.



