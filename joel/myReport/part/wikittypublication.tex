\section{Wikitty Publication}

Wikitty publication est le sous projet sur lequel j'ai été amené à travailler
durant mon stage.

On parle du projet du but du projet
a quoi c'est censé servir

découpage en sous partie, lister les fonctionnalités dans l'ensemble ?
parler du sous découpage que c'esr sensé bien partimenter le taff et tout le
tintouint.



Cette partie du rapport, nous parlerons du sous projet de wikitty: wikitty
publication sur lequel j'ai été amené à travailler durant mon stage. 
Le travail sur ce sous projet à été découper en plusieurs sous partie plus ou
moins indépendantes. 

Dans son ensemble Wikitty publication est un projet basé sur wikitty, dont le
but est de pouvoir d'avoir des applications stockées au sein d'un wikitty et
via une interface web de faire en sorte que ces applications soient directement
éxécutées. 

Wikitty publication une plateforme de stockage exécution d'application
d'application qui peut se





Code Lutin souhaite se lancer dans la création d'une plate-forme Web permettant à des développeurs
et aux utilisateurs de l'application de la modifier aussi simplement qu'elle est utilisée.
Inspirée du concept wiki, cette plate-forme devra permettre de structurer des documents, des
modèles ou des formulaires à l'aide d'un langage spécifique simplifié (semblable à JSP). Cette
approche est celle mise en œuvre dans XWiki, mais cette solution n'est pas adaptée


L'objectif du stage est de partir d'une preuve de concept (déjà réalisée) qu'il faudra reprendre et
compléter pour intégrer notamment les mécanismes d'authentification et d'administration afin
d'arriver à une version utilisable. Comme il s'agit d'une application Web, Struts 2 sera utilisé côté
serveur conjointement avec une base de données Wikitty.
La réalisation de ce projet implique notamment de se pencher sur la réalisation d'un script-engine
(définit dans la JSR 223) et la réutilisation de ceux existants. Cela permettra d'intégrer les différents
langages qui pourront être utilisés pour décrire le contenu du wiki.
Par ailleurs, afin que le système soit compatible avec les pratiques usuelles du développement
(gestion de version, release), il faudra réaliser un outil de synchronisation qui permettra d'extraire le
contenu du wiki de façon structurée, de le modifier puis de publier à nouveau (éventuellement vers
une autre instance du wiki).
Enfin, il faudra produire une documentation de la solution ainsi que travailler à une intégration
Maven, l'outil de construction utilisé chez Code Lutin.





on va mettre le planning du travail et comment ça a été fait
donc découpage en module que on va présenter dans les parties correspondantes
l'organisation avec des tickets
les phases de travail
présentation pendant les réunion dev qui permettent de mettre en exerbe des
points à améliorer.




Ici coller les specs



\subsection{Externalize}




\subsection{Publication}

Amélioration du script engine qui exécute d'autre script etc.

