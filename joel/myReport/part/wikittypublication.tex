\section{Wikitty Publication}

Wikitty publication est le sous module Wikitty sur lequel j'ai été amené à 
travailler durant mon stage. Mon travail s'est articulé autour de plusieurs
partie à l'intérieur de module, avant de rentrer dans le détail, il me faut 
présenter le module Wikitty Publication.

\subsection{Concept}

Wikitty publication est un projet basé sur wikitty, dont le but est de permettre
la création d'application web et leur exécution directement à l'intérieur de
Wikitty Publication. Publication se présente comme un ensemble de page web qui 
permettent le rendu des wikitty dans un navigateur, ainsi que leurs modifications,
comme dans un wiki. On distingue des extensions particulières, une extension
pour les data comme les images et une extension pour le code applicatif.

Les applications doivent pouvoir être développées dans plusieurs langage, pas 
dans un langage spécifique à publication, des langages tel que:
\begin{itemize}
\item javascript
\item groovy
\item jruby
\item jython
\end{itemize}

Ce module doit permettre le développement des applications naturellement 
en développant sur un système de fichier, puis l'importation des applications
directement dans Wikitty Publication.

Un des objectifs de Wikitty Publication est aussi d'avoir un cœur métier 
écrit en wikitty, qui manipulerait des données d'un autres WikittyService, ainsi on
pourrait avoir une mutualisation du code métier et des données spécifique 
pour chaque client, utilisateur de Wikitty Publication.

Wikitty publication s'appuie sur le scripting pour l'exécution du code contenu 
dans les wikitty.


\subsection{Scripting}

Le scripting permet l'éxécution de langage de script par un autre langage, dans
notre cas ici, le java qui peut interpréter de langage comme javascript.

Le langage de script est interprété par un script engine tel que le concept est 
défini dans la JSR-223 \footnote{JSR-223 Scripting for java: http://www.jcp.org/en/jsr/detail?id=223}. 
Un script engine permet l'interprétation et l'exécution
du langage pour lequel il est défini, mais il permet aussi d'insérer des éléments
de code provenant de l'environnement qui l'exécute. Par exemple on peut initialiser
une variable javascript avec le contenu d'une variable java, et on peut appeler 
des méthodes java sur des variables java dans le javascript.

Ce mécanisme est un binding, le script engine pour les symboles qu'il ne connait
pas, avant de renvoyer une erreur, il va chercher dans une map dit de binding 
pour trouver l'objet qui correspond au symbole, et l'interpréter en java pour
insérer le résultat dans le code.

Exemple :

un objet personne dans la map binding, en clé la chaine "personneBob" et en valeur 
l'objet, avec en attribut un nom égal à "Paulson" et un attribut prénom égal à
"Robert".

Un morceau de code js qui sera exécuté:

\lstset{ %
language=JAVA,                % the language of the code
basicstyle=\footnotesize,       % the size of the fonts that are used for the code               % where to put the line-numbers
  % the size of the fonts that are used for the line-numbers
                  % the step between two line-numbers. If it's 1, each line 
keywordstyle=\color[rgb]{,0,1},
                      % will be numbered
numbersep=5pt,                  % how far the line-numbers are from the code
showspaces=false,               % show spaces adding particular underscores
showstringspaces=false,         % underline spaces within strings
showtabs=false,                 % show tabs within strings adding particular underscores
tabsize=2,                      % sets default tabsize to 2 spaces
captionpos=b,                   % sets the caption-position to bottom
breaklines=true,                % sets automatic line breaking
breakatwhitespace=false,        % sets if automatic breaks should only happen at whitespace
title=\lstname,                 % show the filename of files included with \lstinputlisting;
                                % also try caption instead of title
escapeinside={\%*}{*)},         % if you want to add a comment with
morekeywords={project,modelVersion,groupId,description,build, plugin,plugins,configuration, applicationName, wikittyServiceUrl, artifactId,serverID,uploadUrl} 
}

\begin{lstlisting}
var result =
"    <h1>Index</h1>\n"+
"    His name was "+ personneBob.getPrenom() + personneBob.getNom()\n";
var ponay ="";


wpContext.setContentType("text/html");
result;
\end{lstlisting}



A l'exécution sera affiché "His name was Robert Paulson", le javascript sera interprété
normalement, en y joignant les chaines récupérées dans les bindings. Là l'exemple
est simple, mais en pratique cela permet de faire beaucoup de chose.

Dans notre cas avec nos bindings (voir les annexes pour les bindings disponible),
on peut dans du code qui se fera évaluer appeler l'évaluator sur un autre contenu.




