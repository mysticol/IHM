\documentclass[12pt,a4paper,utf8x]{report}
\usepackage [frenchb]{babel}

% Pour pouvoir utiliser 
\usepackage{ucs}
\usepackage[utf8x]{inputenc}

\usepackage{url} % Pour avoir de belles url
\usepackage {geometry}

% Pour mettre du code source
\usepackage {listings}
% Pour pouvoir passer en paysage
\usepackage{lscape}

% Pour pouvoir faire plusieurs colonnes
\usepackage {multicol}
% Pour crééer un index
\usepackage{makeidx}
\makeindex

% Pour gérer les liens interractifs et les signets Acrobat
\usepackage{hyperref}
\hypersetup{
pdftitle={titre de mon document},
pdfauthor={Nom de l'auteur},
pdfsubject={Sujet du document},
pdfkeywords={les mots clefs},
bookmarks, % Création du signet
pdfstartview=FitH, % Page de la largeur de la fenêtre
colorlinks=true, % Liens en couleur
linkcolor=black, 	
anchorcolor=black, 	
citecolor=black, 	
filecolor=black, 	
menucolor=black,
runcolor=black,
urlcolor=black, 	
frenchlinks=black,
bookmarksnumbered=true, % Signet numéroté
pdfpagemode=UseOutlines, % Montre les bookmarks.
bookmarksopen =true,
}

% Pour afficher la bibliographie, mais pas nottoc (Table of Contents), notlof (List of Figures) ni notlot (List of Tables)
\usepackage[notlof, notlot]{tocbibind}


% Pour les entetes de page
% \usepackage{fancyheadings}
%\pagestyle{fancy}
%\renewcommand{\sectionmark}[1]{\markboth{#1}{}} 
%\renewcommand{\subsectionmark}[1]{\markright{#1}} 

% Pour l'interligne de 1.5
\usepackage {setspace}
% Pour les marges de la page
\geometry{a4paper, top=2.5cm, bottom=3.5cm, left=1.5cm, right=1.5cm, marginparwidth=1.2cm}

\parskip=5pt %% distance entre § (paragraphe)
\sloppy %% respecter toujours la marge de droite 

% Pour les pénalités :
\interfootnotelinepenalty=150 %note de bas de page
\widowpenalty=150 %% veuves et orphelines
\clubpenalty=150 

%Pour la longueur de l'indentation des paragraphes
\setlength{\parindent}{15mm}



%%%% debut macro pour enlever le nom chapitre %%%%
\makeatletter
\def\@makechapterhead#1{%
  \vspace*{50\p@}%
  {\parindent \z@ \raggedright \normalfont
    \interlinepenalty\@M
    \ifnum \c@secnumdepth >\m@ne
        \Huge\bfseries \thechapter\quad
    \fi
    \Huge \bfseries #1\par\nobreak
    \vskip 40\p@
  }}

\def\@makeschapterhead#1{%
  \vspace*{50\p@}%
  {\parindent \z@ \raggedright
    \normalfont
    \interlinepenalty\@M
    \Huge \bfseries  #1\par\nobreak
    \vskip 40\p@
  }}
\makeatother
%%%% fin macro %%%%

%Couverture 

\title
{
	\normalsize{Rapport HADL\\
	Université de Nantes\\
	2010-2011}\\
	\vspace{15mm}
	\Huge{Rapport HADL}
}
\author{DEJEAN Charles, POTTIER Vincent\\
	\vspace{45mm}
}


\begin{document}

\maketitle
%Remerciements

Je tiens à remercier :
et on met la liste des personnes que l'on remercie. Toto, tutu, titi. et on met la liste des personnes que l'on remercie. Toto, tutu, titi.et on met la liste des personnes que l'on remercie. Toto, tutu, titi.et on met la liste des personnes que l'on remercie. Toto, tutu, titi.


Et on met la liste des personnes que l'on remercie. Toto, tutu, titi.et on met la liste des personnes que l'on remercie. Toto, tutu, titi.et on met la liste des personnes que l'on remercie. Toto, tutu, titi.et on met la liste des personnes que l'on remercie. Toto, tutu, titi.et on met la liste des personnes que l'on remercie. Toto, tutu, titi.

%\clearpage

\tableofcontents
\clearpage

% Pour avoir un interligne de 1,5
\begin{onehalfspace}

\chapter{Présentation succinte de l'outil HADL}

\section{Comment utiliser l'outil}

Pour commencer, il faut récupérer l'archive du projet, et la dézipper. Le code source complet se trouve dans le directory 'src'.
Il est possible d'importer ce projet en tant que projet Eclipse.

Une fois le projet ouvert dans un IDE, il suffit d'exécuter la classe
\emph{'DescriptionLangage'} située dans le package
\emph{'hadl.m0.descriptionLangage'}.

Une fois cela fait, un menu s'affiche dans la console, offrant des choix de
gestion de configuration.

\section{Architecture globale}
\subsection{Le modèle abstrait}
Les classes du modèle abstrait sont dans les packages hadl.m2.*. Il y a le
composant, la configuration, le composant simple et le connecteur.
\subsection{L'implémentation concrète}
L'implémentation concrète se trouve sous les packages hadl.m1.*. Elle contient
le connecteur RPC, le composant Client, la configuration Serveur. Cette dernière
comprend les connecteurs ClearenceRequest, SecurityQuery et SQLQuery, ainsi que
les composants ConnectionManager, DataBase et SecurityDB.
\subsection{Le langage de description}
Il est contenu dans le package hadl.m0.descriptionLangage. Il contient le
fichier .xml décrivant la configuration CS, le fichier .dtd qui permet de
vérifier si le .xml de description est correct syntaxiquement correct et la
classe \emph{DescriptionLangage}, qui gère le processus d'instanciation des
classes.

\section{Choix d'implémentation}
\subsection{Pattern Oberver}
Nous avons utilisé le pattern oberver pour gérer l'envoie de message entre les
différents éléments composant d'une configuration.

Un composant est Observable c'est-à-dire qu'il peut envoyer un message qui sera
notifié à tous les objets Observateurs qui l'observent.

Une configuration, quant à elle, est à la fois Observable et Observateur.

Ainsi, quand un composant veut communiquer, il le fait savoir à la configuration
parent qui prend alors en charge l'acheminement du message.

\subsection{Pattern Composite}
Nous utilisons un pattern composite au niveau du composant, du composant simple
et de la configuration.

Une configuration et un composant simple dérivent de la classe composant, car
ils ont les mêmes fonctionnalités basiques que le composant, c'est-à-dire
d'avoir un nom, des contraintes et des propriétés. Et une configuration est
composée de composants (entre autre).

\clearpage


% Pour finir l'interligne de 1,5
\end{onehalfspace}

%----------------------------------------
% Pour la bibliographie
%----------------------------------------
% Citer tous les ouvrages/références
% \nocite{*}
% Trier par ordre d'apparition
% \bibliographystyle{unsrt}
% Pour le style de la biblio
% \bibliographystyle{plain.bst}
% Ecrire la biblio ici
% \bibliography{biblio}

\printindex

\appendix


\end{document}
