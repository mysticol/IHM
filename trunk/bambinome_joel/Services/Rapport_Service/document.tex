\documentclass[11pt,a4paper]{article}
\usepackage{ifpdf}
\usepackage[utf8]{inputenc}
\usepackage[francais]{babel}
\usepackage[T1]{fontenc}
\usepackage[nottoc, notlof, notlot]{tocbibind}
\usepackage[unicode=true,pdftex,colorlinks=true,linkcolor=black,urlcolor=black,citecolor=black]{hyperref}
\usepackage{natbib}
\usepackage{graphicx}

\parindent 0.8cm
%\setlength{\parskip}{0.5em plus 0.2em minus 0.2em}

\title{Projet Service : ALMA' Turn-keyWeek-End}
\author{Anthony \textsc{Caillaud} Manoël \textsc{Fortun}}
\date{\today}
\ifpdf
\pdfinfo {
/Author (Anthony Caillaud Manoël Fortun)
/Title (Projet Service : ALMA' Turn-keyWeek-End)
/Subject (Projet Service : ALMA' Turn-keyWeek-End)
/Keywords ()
/CreationDate (D:20100329212218)
}
\fi


\begin{document}

\maketitle


\clearpage
\tableofcontents
\clearpage
\section{Introduction}

Dans le cadre du module Service dans lequel nous avons étudié tous les
mécanismes et tous les éléments d'une architecture basée sur les services, nous
avons dû mettre en place une SOA complet. Ce SOA doit permettre à un utilisateur
de réserver plusieurs éléments pour un week-end pour deux personnes. Ces
éléments à réserver sont le voyage jusqu'à la destination, deux tickets pour
une manifestation, une chambre pour une nuit dans un hôtel et un diner dans un
restaurant. Afin d'obtenir une SOA fonctionnelle, nous avons d'abord mis en
place notre base de données et déterminé les tables nécessaires. Ensuite, nous
avons fixé les services nécessaires pour le bon déroulement de l'application, du
choix de l'utilisateur jusqu'aux différentes réservations. Une partie
importante de cette architecture sont les interfaces. L'une
d'entre-elle qui sera la façade de notre application pour l'utilisateur et
l'autre qui sera le back office permettant la gestion des entrées dans les
tables vous seront présentées. Enfin la génération du bon de réservation à
partir des données choisies par l'utilisateur à l'aide d'une transformation
XSLT permettra à celui-ci d'obtenir ce document en dans un fichier pdf.


\section{Définition de la base de données}
Pour cette architecture, nous avons utilisé une base de données comprenant les
différentes tables nécessaires au bon fonctionnement de l'application. Notre
base de données contient donc onze tables. 

Tout d'abord, il y cinq tables qui contiennent toutes les entrées représentant
les choix proposés à l'utilisateur. Ces tables sont les suivantes :\\

\begin{itemize}
  \item La table LOCALISATION qui permet de choisir un pays et une ville,
  \item La table TYPE MANIFESTATION qui permet de choisir le type de
  manifestation,
  \item La table MANIFESTATION qui permet de choisir une manifestation d'un
  certain type et qui a lieu dans le pays et la ville choisis,
  \item La table HOTEL qui permet de choisir un hôtel présent dans la ville
  choisie,
  \item La table RESTAURANT qui permet de choisir un restaurant présent dans la
  ville choisie.\\
\end{itemize}
 
Ensuite, cinq autres tables permettent l'enregistrement des différentes
réservations effectuées par l'utilisateur. Ces tables sont les suivantes :\\

\begin{itemize}
  \item La table VOYAGE qui enregistre la réservation du voyage entre la ville
  de départ et la ville d'arrivée,
  \item La table RESERVATION MANIF qui enregistre la réservation des deux tickets
  pour la manifestation choisie,
  \item La table RESERVATION HOTEL qui enregistre la réservation d'une chambre pour
  une nuit dans l'hôtel choisi,
  \item La table RESERVATION RESTAURANT qui enregistre la réservation d'un
  diner au restaurant,
  \item La table RESERVATION qui enregistre toutes les données concernant les
  réservations du week-end.\\
\end{itemize}

Enfin, la dernière table est la table CLIENT qui permet l'enregistrement du nom
et du prénom de l'utilisateur voulant effectuer ce week-end.


\section{Les différents services}
\subsection{La recherche des disponibilités}
\subsection{Les réservations}

\section{Les interfaces}
\subsection{Définition de l'interface}
\subsubsection{L'interface JSP}
\subsubsection{GoogleMaps}
\subsection{Le back office}

Principe de fonctionnement role du backoffice
gestion fine de la bd
pas d'aspect service dans le bacj office

approche: utilisation de symphony

utilisation de swing et hibernate parce que facile

\subsubsection{architecture}

swing hibernate
bean 
fonctionnement

\subsubsection{hibernate}
Utilisation d'hibernate avec génération automatique de certain élément 
principe d'hibernate

\subsubsection{Swing}
Utilisation de swing tout ça

tjable organiser les donnée
Interfacequi évolue
le moins possible de vue
faire simple.

\section{Génération du bon de réservation}

\section{Conclusion}

Utilisation de netbean pas particulièrement très agréable. 
Projet très ambitieux pour un projet de module avec trop d'aspect étranger au module 
Pas assez de temps
Une partie du projet sur lequel on ne peut avoir aucun support
prof de merde
gros connard


\end{document}
