\documentclass[11pt,a4paper]{article}
\usepackage{ifpdf}
\usepackage[utf8]{inputenc}
\usepackage[francais]{babel}
\usepackage[T1]{fontenc}
\usepackage[nottoc, notlof, notlot]{tocbibind}
\usepackage[unicode=true,pdftex,colorlinks=true,linkcolor=black,urlcolor=black,citecolor=black]{hyperref}
\usepackage{natbib}
\usepackage{graphicx}

\parindent 0.8cm
%\setlength{\parskip}{0.5em plus 0.2em minus 0.2em}

\title{Projet HADL}
\author{Anthony \textsc{Caillaud} Manoël \textsc{Fortun}}
\date{\today}
\ifpdf
\pdfinfo {
/Author (Anthony Caillaud Manoël Fortun)
/Title (Projet HADL)
/Subject (Projet HADL)
/Keywords ()
/CreationDate (D:20100329212218)
}
\fi


\begin{document}

\maketitle


\clearpage
\tableofcontents
\clearpage
\section{Introduction}

Dans le cadre de notre module composant dans lequel nous avons pu étudier les modèles à composant et leur fonctionnement, nous avons du modéliser et écrire notre propre langage d'architecture à composant. Ce projet ne consistait pas seulement en la définition d'un langage d'architecture à composant, mais aussi l'implémentation en langage objet d'un moteur d'architecture à composant. Ce rapport présente donc le travail que nous avons pu effectuer sur cet HADL(home architecture definition langage), dans un premier temps au travail de modélisation des concepts composants (meta modélisation), l'implémentation du modèle. Puis ensuite l'application du modèle à composant (modélisation) à travers l'exemple donné en cours. Puis nous détaillerons le meta meta modèle que nous avons pu définir pour notre HADL.


 
\section{Description du méta-modèle HADL}
\subsection{Définitions}
L'architecture logicielle à base de composants est consitutés de trois
principaux éléments : des comosants, des configurations et des connecteurs.\\

\subsection{ Eléments principaux}

% - Def Composant
 Un composant est une unité de composition qui spécifie
une ou plusieurs fonctionnalités. Celui-ci peut être déployé indépendemment, il s'agit de la brique élémentaire.\\

%- Def Configuration
Une configuration est un ensemble de composants et de connecteurs, il définit la
façon dont ils sont reliés entre eux. Cette unité de l'architecture logicielle
à base de composants est nécessaire aux bonnes liaisons entre composants. Elle
permet de savoir si ces liaisons relatives aux connecteurs permettent une
communication correcte. C'est la configuration qui gère tous les composants et
tous les connecteurs qui la composent.\\

%- Def Connecteur
Un connecteur explicite est une unité effectuant une liaison entre des
composants ou des configurations. Les connections proprement dites s'effectuent
grâce à des Rôles (Rôles From et Rôles To) et une ''Glue'' représente la
fonction modifiant les informations transférées. Un adaptateur entre deux composants\\

\subsection{Eléments secondaire}

%-Def Interface 
L'interface d'un composant, connecteur ou configuration est l'ensemble des éléments fourni pour l'intéraction avec le composant, connecteur ou configuration. Il s'agit plus d'un concept que d'un élément concret. Une interface est généralement constituée de port et service.

%- Def Port
Un port est un point d'entré ou de sortie d'un composant ou configuration, il pourrait s'apparenter à un attribut d'une classe si on fait le rapprochement entre modèle composant et modèle objet.

%- Def  Service
Un service est une opération disponible auprès d'un composant, il y a des services fournit par un composant et des services requis pour son fonctionnement. Un service peut être assimilé à une méthode toujours dans le parallèle composant/objet. Un service peut être relié à des ports ce qui signifie que certain port seraient les entrés du service et d'autres ports seraient la ou les sorties du service.

%-Def Propriétés
Les propriétés sont des éléments appartenant aux configurations ou composants, les propriétés peuvent définir une particulatité du comportement, ou un élément que dont on peut se servir, comme un attribut de classe.

%-Def Contraintes
Les contraintes sont des éléments importants, puisque elles définissent ce dont ont besoin les composants ou configuration pour fonctionner,  par exemple la présence d'une librairie particulière.

%-Def Roles
Ce concept est propre aux connecteurs, il y a les roles from et role to qui constituent l'interface des connecteurs. Leurs entrés et leurs sorties.

%-Def Lien
Les liens sont des éléments qui permettent de relier des composant entre eux, ou une configuration à un composant, on distinguera deux types de liens:
-Les attachements qui sont des liens entre composants, lien qui inclut un connecteur
-Les binding qui relie les ports de configuration aux port de composant, lien qui n'inclut pas de connecteur


\subsection{Analyse}

Lien= attachement= binding= connecteur

port service

Parler des fonctions utile




\subsection{Diagrammes}

diagramme de l'implémentation

Quelques explications



\section{Description du système Client/Serveur}
\subsection{Définitions}
%- Def Serveur
%- Def Client
\subsection{Analyse}
\subsection{Diagrammes}

\section{Description du méta-méta-modèle HADL}
\subsection{Définitions}
%- Def Meta-Entité
%- Def Meta-Relation
\subsection{Analyse}

mettre que tout les éléments ne sont pas mis pour pas surcharger le diagramme et que globalement sinon tout les élement du m2 sont des entité et compose d'aytre trop

\subsection{Diagrammes}

\section{Conclusion}



\end{document}
