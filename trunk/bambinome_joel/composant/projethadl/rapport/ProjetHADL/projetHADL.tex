\documentclass[11pt,a4paper]{article}
\usepackage{ifpdf}
\usepackage[utf8]{inputenc}
\usepackage[francais]{babel}
\usepackage[T1]{fontenc}
\usepackage[nottoc, notlof, notlot]{tocbibind}
\usepackage[unicode=true,pdftex,colorlinks=true,linkcolor=black,urlcolor=black,citecolor=black]{hyperref}
\usepackage{natbib}
\usepackage{graphicx}

\parindent 0.8cm
%\setlength{\parskip}{0.5em plus 0.2em minus 0.2em}

\title{Projet HADL}
\author{Anthony \textsc{Caillaud} Manoël \textsc{Fortun}}
\date{\today}
\ifpdf
\pdfinfo {
/Author (Anthony Caillaud Manoël Fortun)
/Title (Projet HADL)
/Subject (Projet HADL)
/Keywords ()
/CreationDate (D:20100329212218)
}
\fi


\begin{document}

\maketitle


\clearpage
\tableofcontents
\clearpage
\section{Introduction}

 
\section{Description du méta-modèle HADL}
\subsection{Définitions}
L'architecture logicielle à base de composants est consitutés de trois
principaux éléments : des comosants, des configurations et des connecteurs.\\

% - Def Composant
 Un composant est une unité de composition qui spécifie
une ou plusieurs fonctionnalités. Celui-ci peut être déployé indépendemment et
peut être composé avec d'autres composants.\\

%- Def Configuration
Une configuration est un ensemble de composants et de connecteurs, il définit la
façon dont ils sont reliés entre eux. Cette unité de l'architecture logicielle
à base de composants est nécessaire aux bonnes liaisons entre composants. Elle
permet de savoir si ces liaisons relatives aux connecteurs permettent une
communication correcte. C'est la configuration qui gère tous les composants et
tous les connecteurs qui la composent.\\

%- Def Connecteur
Un connecteur explicite est une unité effectuant une liaison entre des
composants ou des configurations. Les connections proprement dites s'effectuent
grâce à des Rôles (Rôles From et Rôles To) et une ''Glue'' représente la
fonction modifiant les informations transférées.\\

\subsection{Analyse}
\subsection{Diagrammes}


\section{Description du système Client/Serveur}
\subsection{Définitions}
%- Def Serveur
%- Def Client
\subsection{Analyse}
\subsection{Diagrammes}

\section{Description du méta-méta-modèle HADL}
\subsection{Définitions}
%- Def Meta-Entité
%- Def Meta-Relation
\subsection{Analyse}
\subsection{Diagrammes}

\section{Conclusion}



\end{document}
