\section{Code Lutin}

\subsection{Présentation de l'entreprise}


Code Lutin est une Société de Service en Logiciel Libre créée en 2002, elle est
implantée à St Sebastien sur Loire, et travail principalement pour des clients
dans la région du grand ouest.

L'entreprise s'est spécialisée autour des technologies Java JEE et UML:
conception, architectures JEE, outils JEE, MDA (Model Driven Architecture), 
développement/migration d’applications JEE, formation.

L'entreprise étant une SSLL elle travaille uniquement avec des outils issue du
logiciel libre, et autant que celà est possible les projets sur lesquelles elle
va être amené à travailler seront des logiciels libres.

Code Lutin offre les services de : 
\begin{itemize}
\item développement de logiciel (forfait ou régie)
\item l’intégration et de la maintenance de systèmes clés en main
\item support
\item conseil et de la veille technologique
\end{itemize}

Services principalement autour des technologies Java.

%rajuter quelque chose sur le mode de fonctionnement ? A base que ce sont les
%employés qui finalement prennent toutes les déscision. 

\subsection{Acteur du monde libre}

Code Lutin se veut acteur du monde du logiciel libre, l'entreprise soutient des
projets du monde libre. Elle est membre de l'april, une association qui à pour
but de promouvoir et défendre le loficiel libre.

L'entreprise est membre de Alliance libre une association nantaises qui regroupe
les entreprises de la région qui travail pour le libre. Le but étant de
promouvoir le logiciel libre dans l'ouest.

L'entreprise fait aussi partit du réseau libre entreprise qui regroupe des
entreprises françaises proches du logiciel livre, et qui partagent des modes de
fonctionnement similaire et les mêmes valeurs.

% section à étoffer


\subsection{Leurs projets}

En plus de faire du logiciel au forfait et en régie, l'entreprise investit une
grande force dans la recherche et dévellopement, en travail sur leurs librairies
qu'ils utilisent dans leurs divers projet. Mais aussi en travaillant sur des
solutions logiciels qui seront d'abord utilisés en interne, puis proposés
ensuite à des clients une fois que la solution aura été finalisée.

Un exemple, c'est Chorem qui est une collection outil de gestion d'entreprise
pour la facturation, la gestion des employés, des contrats, etc. Une collection
d'outil qui à vu le jour parce Code Lutin ne trouvait pas de logiciel de gestion
d'entreprise, libre. Lima est un autre exemple, pendant libre des solutions de
gestion de comptabilités existantes.
