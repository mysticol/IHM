\section*{Introduction}

Ce rapport présente le stage que j'ai effectué dans le cadre de mon master 2
ALMA, stage effectué chez Code Lutin. J'ai effectué mon stage chez Code Lutin, une
entreprise qui fait du logiciel libre. J'avais choisi justement cette entreprise
pour ça, l'occasion de faire du logiciel libre. Pour moi c'était une chance 
d'apprendre de bonne pratique, de progresser techniquement. Aussi de contribuer au 
monde du libre et d'en apprendre plus sur les valeurs du libre.

Ce rapport présente donc mon travail sur Wikitty Publication, module de Wikitty
un projet libre initié par Code Lutin. Le concept est assez simple et très 
intéressant, celui d'une application que l'on peut modifier comme on l'utilise :
comme dans un Wiki. On y accèderait dans un navigateur simplement, et on la modifierait aussi
dans le navigateur.

Une application qui du coup pourrait elle même se modifier, un stockage en objet
du code de l'application et des données manipulées, de sorte que l'application 
puisse accéder aux deux de la même façon.

Mon stage a débuté avec le prototype de l'application qui était fonctionnel,
mais pas utilisable en dehors du cadre prototype, mon travail a été de travailler
sur ce prototype pour le rendre utilisable pour le développement d'application.

Je présenterais donc dans ce rapport l'entreprise pour laquelle j'ai travaillé,
le projet Wikitty, le concept de Wikitty publication, ensuite les différentes
fonctionnalités sur lesquelles j'ai travaillé sur ce module.
Enfin je conclurais sur mon stage et les travaux que j'ai effectué.

