\section{Wikitty Publication Synchronisation}

La partie synchronisation de Wikitty Publication concerne des fonctionnalités
destinées à importer une application auprès d'un WikittyService. Mais pas
seulement, on doit pouvoir synchroniser deux WikittyService quelque soit son
implémentation, que ce soit un client cajo, hessian, un client base de donné ou
même un WikittyService sur file system.

Cette approche de la synchronisation permet une grande flexibilité de
l'application et permet à terme la synchronisation de wikitty entre WikittyService,
 et pas seulement ceux concerné par l'aspect publication.

Cette approche implique la création d'une implémentation spécifique d'un wikitty
service pour stocker des wikitty en fichier et restaurer des fichiers en
wikitty.


\subsection{Correspondance Fichiers/Wikitty}

Un fichier est défini par un nom, une extention, un contenu et un chemin.
dans Wikitty Publication les fichiers sont convertis en fonction de leurs type
en objet wikitty. Les fichiers sources sont convertit en WikittyPubText et les
fichiers binaires(eg image, etc) en WikittyPubData.

Ces deux types de wikitty pourrons être dénommés dans la suite sous le nom
WikittyPub. 

Les deux types d'objet ont les mêmes attributs:
\begin{itemize}
\item Name: correspondant au nom du fichier
\item MimeType: correspondant au type mime du contenu, qui donnera l'extension
\item Content: le contenu binaire pour pour les PubData et textuel pour les
PubText
\end{itemize}

Le mimetype détermine le contenu, le mime type est un élément très important,
la détermination du mimetype d'un fichier n'est pas laissé à une librairie,
elle est déterminer par un mimeTypeHandler qui contient un mapping défini
qui permet de connaitre l'extension du fichier et si il s'agit d'un 
WikittyPubText ou WikittyPubData. Le mimeType est aussi très important dans 
le processus d'évaluation d'un WikittyPub.

A ces objets wikitty on associe un wikittyLabel, c'est un objet qui peut
contenir un ensemble de label différent, un label une chaine de texte.
Par exemple le label : \emph{src.org.chorem.entities} sert ici pour contenir le
chemin menant au fichier sur le file system. Un WikittyPub peut avoir un ou
plusieurs de label, pour les WikittyPub cela indiquera qu'ils appartiennent à
plusieurs arborescences.

Pour enregistrer un wikitty en fichier, il suffit de construire l'arborescence
des dossiers en fonction du label du WikittyPub. Le nom du fichier est donné
par l'attribut nom du WikittyPub, son contenu aussi et l'extension par le mimeType.

Pour assurer la restauration d'un WikittyPub qui aura été sauvegardé en fichier,
il faut enregistrer en plus des informations supplémentaire. Comme tout Wikitty,
les WikittyPub possèdent un numéro de version ainsi qu'un id qui les identifient
de façon unique.

Il faut donc enregistrer ces informations, de plus il faut que le wikitty
service sur file système gère les modifications sur wikitty et donc modifie la
version des wikitty si besoin.

La solution qui à été trouvée pour celà est les fichiers de propriété dans un
dossier \emph{.wp/}. 

On distinguera deux fichiers de propriétés pour les informations, un qui 
conservera les id des wikitty lié à leur nom de fichier. Et un autre fichiers de
propriété qui conservera un checksum, la version et les id aussi. En sus de
la sauvegarde des "meta information" des wikitty, ce fichier de propriété
conservera le nom du label courant, rendant immédiat la restauration (pas
besoin de traitement complexe sur les noms de fichiers/dossiers).

On conserve les id dans un premier fichier puisque cela permet simplement de 
récupérer l'ensemble des id et leurs noms de fichier lié, sans avoir besoin de 
faire le tri parmi toutes les propriétés enregistrées. On converse l'id aussi 
dans un autre fichier de propriété, à défaut d'avoir un systeme d'enregistrement
clé/valeur bidirectionnel. 

La propriété checksum sera utilisée pour enregistrer la somme de contrôle de 
l'objet lors de son enregistrement, pour plus tard, savoir si celui ci à été
modifié depuis lors.

Exemple pour un contenu de file system:
\begin{verbatim}
 +racine
 |script.js
 |test.java
 |image.png
 |+.wp
 ||ids.properties 
 ||meta.properties 
 |+directory2
 ||script3.js
 ||+.wp
 |||meta.properties
 |||ids.properties 
 ||+directory3
 |||truc.js
 |||+.wp
 ||||meta.properties
 ||||ids.properties 
 |+directory22
 ||machin.png
 ||+.wp
 |||versions.properties
 |||ids.properties 
\end{verbatim}


Exemples de fichiers de propriétés:
\begin{verbatim}
racine/.wp/meta.properties:

current.label=racine
script.js.version=0.2
version.scripttut.js=1.5
version.image.png=12.0
checksum.script.js=c48e20c27c00779be486ecac8f03c8e7
checksum.scripttut.js=bffba35cafca7ad1ac9eee1b643579aa
checksum.image.png=c48e20c27c00779be486ecac8f03c8e7
id.image.png=97410079-2c0c-47db-958e-8309d7147e2e
id.test.java=96fe0754-5ba5-47d5-a763-0adf0e839e47
id.script.js=d5a2597f-c44f-4ed6-a16e-01371ec6d2d0
\end{verbatim}

\begin{verbatim}
racine/.wp/ids.properties:

97410079-2c0c-47db-958e-8309d7147e2e=image.png
96fe0754-5ba5-47d5-a763-0adf0e839e47=test.java
d5a2597f-c44f-4ed6-a16e-01371ec6d2d0=script.js
\end{verbatim}

\subsection{Un wikitty service file system}

Un WikittyService implémenté pour un système de fichier doit pouvoir enregistrer
des wikitty en fichier et inversement. Pour se faire il se servira des fichiers
de propriété précédemment exposé, et des conversions telles qu'elles le sont
proposées.

En plus de la sauvegarde, restauration et suppression, ce type de WikittyService
doit, c'est la base du système de synchronisation, gérer les recherches par
critère des wikitty comme si il était stocké de façon classique dans une base.

Pour correctement gérer ces options de sauvegarde, restauration, gestion de
version des wikitty, pour tout appel à la recherche ce wikitty service indexe
l'ensemble des fichiers:
\begin{itemize}
\item vérifie si il y a des nouveaux fichiers en comparant les valeurs dans les
fichier de propriétés "id.properties". Si il ya des nouveaux fichiers il créé
des id, calcul les sommes de contrôle et enregistre le tout dans les fichiers de
propriétés.
\item il vérifie les sommes de contrôles des fichiers existants, et incrémente la
version pour ceux qui ont changé, et il met à jour la somme de contrôle
\item il cherche si des fichiers ont été supprimé, et le cas échéant enlève les
infos concernant le wikitty correspondant au fichier supprimé
\end{itemize}

Avec ces fonctions le wikitty service file system est "complet" pour la
synchronisation, et pour tout autre utilisation.

Le mécanisme qui indexe créer automatiquement les fichiers de propriété
nécessaire, ce qui implique que une arborescence de fichier peut être converti
en wikitty de cette façon, nul besoin d'initialiser un espace de travail en en
le créant à partir de wikitty, tout dossier peut devenir "une base wikitty".

\subsection{Fonctionnalités de synchronisation}

\subsubsection{Sync}

La fonctionnalité sync permet de transférer l'ensemble des wikitty ciblés par 
l'uri, d'un service wikitty à un autre. Son fonctionnement doit est similaire à
la commande linux "rsync".

Quelques options on été reprise:
\begin{itemize}
\item Recursion pour savoir si l'on s'occupe des sous labels du label ciblé.
\item Update, qui permet de mettre à jour la cible et d'y envoyer les nouveaux 
wikitty. Par défaut cette option est active, et est désactivée lorsque 
les autres options (delete ou existing) sont choisis.
\item Existing qui est un update mais sans l'envois des nouveaux fichiers, on
envois juste ce qui à été mis à jour et qui existe sur le wikitty service cible.
\item Delete pour supprimer dans le WikittyService cible, ce qui n'existe plus
dans le WikittyService origine.
\end{itemize}

La suppression n'est pas une vraie suppression, elle se contente de supprimer le 
label ciblé du wikitty. L'éventuel suppression définitive du wikitty si celui ci
se retrouve sans label est à la charge du service.

La partie protocole de l'uri permet de retrouver la bonne implémentation pour 
communiquer avec le WikittyService désigné.

La commande:

\verb!wp sync [--norecursion] [--delete|--existing] [URI orgirine] [URI cible]!

Avec URI sous forme: 
\begin{itemize}
\item file:///truc/machin/\#label
\item hessian://www.adresse.com:8827/etc/etc\#label
\item cajo://www.adresse.com:8827/etc/etc\#label
\end{itemize}

Évidement le path et le label pour une uri de type file détermine l'espace de
travail wikitty. eg: \emph{/truc/machin/label} est le repertoire racine

Les labels de l'uri cible et origine peuvent être différent, ce qui signifie que
les labels ne sont pas transmit, le wikitty sur la cible est modifié mais pas 
ses labels.

Dans le cas de wikitty qui n'existe pas dans la cible, on remplacera le label
origine qui a permit de trouver ce wikitty par le label cible. 

\subsubsection{Commit/update}

Les fonctionnalités commit/update sont des alias pour la commande sync.
L'utilisation de ces commandes implique la présence d'un espace de travail
précédemment initialisé/utilisé avec la commande sync.

Les commandes commit et update permette de ne pas avoir besoin de préciser le
WikittyService cible de la synchronisation. On va synchroniser avec le dernier
WikittyService ciblé. Pour ce fonctionnement le la commande sync enregistre
au niveau du répertoire de travail du Service sur File System, un fichier
de propriété contenant l'adresse du dernier WikittyService avec lequel on l'a 
synchronisé. 


Un commit sera équivalent à un sync de l'espace de travail local vers un autre
wikitty service. 

Un update sera équivalent à un sync d'un autre wikitty service vers l'espace de
travail local.

La commande:

\verb!wp [update|commit] [--norecursion] [--delete|--existing] [label] [URI file]!

L'URI file est optionnelle, si pas précisée on va commit/update le dossier 
courant.

Pour le commit/update, on doit forcément préciser le label pour le wikitty 
service cible, cela pour permettre de commit/update au bon endroit.
La commande supporte évidement les mêmes options que pour une commande sync, en
toute logique.


\subsection{Fonctionnement de sync}


Une synchronisation entre deux WikittyService se fait en comparant les wikitty
qu'ils ont chacun, wikitty qui possèdent soit des labels précis ou des labels
commençant par une chaine, fonction de l'option récursion.

On distinguera les wikitty qui existent dans les deux services, ceux présent
uniquement dans le wikitty service cible et ceux uniquement sur le wikitty
service d'origine.

Ensuite en fonction de l'option de synchronisation ces différentes collection de
wikitty sont utilisé.

Dans le cas d'une uptate tout les wikitty qui existe uniquement dans l'origine
son envoyé, et ceux présent dans les deux sont mis à jour.

Dans le cas d'un existing seulement ceux présent dans les deux services sont mis à jour.
Et dans le cas d'un delete on va supprimer les labels des wikitty seulement
présent sur la cible.

