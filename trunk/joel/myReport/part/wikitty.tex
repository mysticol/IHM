\section{Projet Wikitty}


\subsection{principe}

Ce projet est une base de données orientée documents et API de persistance pour
Java. Il permet d'enregistrer dans une base clé/valeur des objets wikitty, la
clé est un id généré qui est unique pour chaque wikitty.

Un wikitty possède des extensions et les extensions possèdent des champs nommés
et chaque champ est une clé à laquelle on associe des valeurs. Et des extensions
peuvent dépendre d'autre extension.

Par exemple une extension personne qui possède deux champs : nom et prénon.
Une autre extension ressource qui possède un champs numéro et qui dépends de
l'extension personne.

Si un wikitty possède l'extension ressource alors ce wikitty possèdera les
champs de l'extension ressource et de l'extension employé, puisque ressource
dépends d'employé.

L'api de wikitty prévois un certain nombre d'extension basique, mais tout
l'intéret est que l'on peut créer ses propres extensions, simplement en écrivant
un modèle uml de classe, qui servira ensuite pour générer les classes java
effectives.

Ensuite il suffit de développer sont applications simplement en utilisant les
classes générées et les passer à un wikitty service pour le stockage sans plus
se poser de question le tout très simplement.

Cette api à été conçu pour supporter les monter en charge et une utilisation
intensive. Elle supporte aussi la recherche par criteria, et par facette.

faire une description ici de criteria et facette.
Mettre un dessin qui montre l'application tel qu'elle peut être déployé.
détailler comment faire ses classes java en uml.


faut parler de la gestion de version des wikitty.


\subsection{existant}

L'application peut être déployé sur serveur avec le protocole cajo, hessian et
d'autre chose.

Wikitty sert de base de donnée pour une partie des applications de l'entreprise,
puisque il permet une grande flexibilité, et à partir du moment ou on peut
modéliser les donnés en objet on peut les stocker en base.


