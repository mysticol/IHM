\section{Projet Wikitty}

Wikitty est un projet important chez code lutin, mon travail à été sur un sous
projet de wikitty, et donc pour le comprendre il est important de présenter
le projet Wikitty.


\subsection{principe}

Ce projet est une base de données orientée documents et API de persistance pour
Java. Il permet d'enregistrer dans une base clé/valeur des objets wikitty, la
clé est un id généré qui est unique pour chaque wikitty.

Un wikitty possède des extensions et les extensions possèdent des champs nommés
et chaque champ est une clé à laquelle on associe des valeurs. Et des extensions
peuvent dépendre d'autre extension.

Par exemple une extension personne qui possède deux champs : nom et prénon.
Une autre extension ressource qui possède un champs numéro et qui dépends de
l'extension personne.

Si un wikitty possède l'extension ressource alors ce wikitty possèdera les
champs de l'extension ressource et de l'extension employé, puisque ressource
dépends d'employé.

L'api de wikitty prévois un certain nombre d'extension basique, mais tout
l'intéret est que l'on peut créer ses propres extensions, simplement en écrivant
un modèle uml de classe, qui servira ensuite pour générer les classes java
effectives.

Ensuite il suffit de développer son application simplement en utilisant les
classes générées et les passer à un wikitty service pour le stockage sans plus
se poser de question le tout très simplement.

Cette api à été conçue pour supporter les monter en charge et une utilisation
intensive. Elle supporte aussi la recherche par criteria, et par facette.

La recherche par criteria permet de rechercher des wikitty en fonction de leurs
extensions et des valeurs des différents éléments de ces extensions.

Les facettes permettent de regrouper les résultats de recherche par criteria. 

Tout wikitty posséde un numéro de version, qui évolue en fonction des
modifications faites dessus. L'api prévois que l'on puisse restaurer une version
ciblé d'un wikitty.
%quand wikitty version sera mieux gérer en parler.

\subsection{existant}

Wikitty sert de base de donnée pour une partie des applications de l'entreprise,
puisque il permet une grande flexibilité. Dès lors ou on peut modéliser les
données en objet on peut les stocker dans avec Wikitty.

Différentes applications ?
Extension, sous projet existant.

Wikitty à été étofé par des sous projet autorisant le déploiement de base
wikitty sur des serveurs avec des protocole Hessian ou Cajo.

Parler de quelques projet se basant dessus.

Objectif concernant ce projet les aspirations de wikitty.
