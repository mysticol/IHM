\section{Wikitty Publication}

Wikitty publication est le sous module Wikitty sur lequel j'ai été amené à 
travailler durant mon stage. Mon travail s'est articulé autour de plusieurs
partie à l'intérieur de module, avant de rentrer dans le détail, il me faut 
présenter le module Wikitty Publication.

\subsection{Concept}

Wikitty publication est un projet basé sur wikitty, dont le but est de permettre
la création d'application web et leur éxécution directement par wikitty
publication. 

Publication se présente comme un ensemble de page web qui permettent le rendu 
des wikitty dans un navigateur, pour les wikitty de type data, comme les images,
et comme le résultat de l'évaluation du code contenu dans les wikitty type text
qui eux contiennent le code des applications.

Les applications doivent pouvoir être developpées dans plusieurs langage, pas 
dans un langage specifique à publication, des langages tel que:
\begin{itemize}
\item javascript
\item groovy
\item jruby
\item jython
\end{itemize}

Ce module doit permettre le dévellopement des applications naturellement 
en devellopant sur un système de fichier, puis l'importation des applications
directement dans wikitty publication.

Un des objectifs de wikitty publication est aussi d'avoir un coeur métier 
écrit en wikitty, qui manipulerait des données d'un autres wikitty, ainsi on
pourrait avoir une mutualisation du code métier et des données spécifique 
pour chaque client, utilisateur de wikitty publication.

Wikitty publication s'appuie sur le scripting pour l'éxécution du code contenu 
dans les wikitty.


\subsection{Scripting}

Le scripting permet l'éxécution de langage de script par un autre langage, dans
notre cas ici, le java qui peut interpréter de langage comme javascript.

Le langage de script est interprété par un script engine tel que le concept est 
défini dans la JSR-223, un script engine permet l'interprétation et l'éxécution
du langage pour lequel il est défini, mais il permet aussi d'insérer des élements
de code provenant de l'environnement qui éxécute. Par exemple on peut initialiser
une variable javascript avec le contenu d'une variable java, et on peut appeler 
des méthodes java sur des variables java dans le javascript.

Ce méchanisme est un binding, le script engine pour les symboles qu'il ne connait
pas, avant de renvoyer une erreur, il va chercher dans une map dit de binding 
pour trouver l'objet qui correspond au symbole, et l'intépréter en java pour
insérer le résultat dans le code.

Exemple :

un objet personne dans la map binding, en clé la chaine "personneBob" et en valeur 
l'objet, avec en attribut un nom égal à "Paulson"

Un morceau de code js qui sera éxécuté:

alert("hello world Mister " + personneBob.getNom());

A l'éxécution sera affiché "hello world Mister Paulson"


%mettre ici ce qui est disponible au niveau des bindings 
%ça va servir aussi pour la doc TODO



