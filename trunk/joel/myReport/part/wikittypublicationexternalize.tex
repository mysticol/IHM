\lstset{ %
language=JAVA,                % the language of the code
basicstyle=\footnotesize,       % the size of the fonts that are used for the code               % where to put the line-numbers
  % the size of the fonts that are used for the line-numbers
                  % the step between two line-numbers. If it's 1, each line 
keywordstyle=\color[rgb]{1,0,1},
                      % will be numbered
numbersep=5pt,                  % how far the line-numbers are from the code
showspaces=false,               % show spaces adding particular underscores
showstringspaces=false,         % underline spaces within strings
showtabs=false,                 % show tabs within strings adding particular underscores
tabsize=2,                      % sets default tabsize to 2 spaces
captionpos=b,                   % sets the caption-position to bottom
breaklines=true,                % sets automatic line breaking
breakatwhitespace=false,        % sets if automatic breaks should only happen at whitespace
title=\lstname,                 % show the filename of files included with \lstinputlisting;
                                % also try caption instead of title
escapeinside={\%*}{*)},         % if you want to add a comment with
morekeywords={project,modelVersion,groupId,description,build, plugin,plugins,configuration, applicationName, wikittyServiceUrl, artifactId,serverID,uploadUrl} 
}



\section{Wikitty Publication Externalize}

La partie Externalize de Wikitty Publication concerne les fonctionnalités 
d'externalisation des wikitty. C'est à dire prendre des wikitty stockés sur un 
système de fichier, via le service correspondant précédemment développé, compiler
ces wikitty et les stocker dans un jar.

L'objectif recherché par cette partie de Wikitty Publication est la performance,
en ayant du code compilé on gagne évidement en rapidité d'exécution des wikitty
compilé. Et du code non modifiable, par exemple dans le cadre de l'utilisation 
multi-contexte, on pourrait avoir un service s'appuyant sur des wikitty stockés 
dans un jar.

Comme pour la synchronisation vers file system, sont concernés par ce mécanisme
uniquement les Wikitty représentable sur file system. Comme les wikittyPubData,
qui ne seront pas affecté par le mécanisme. Et les wikittyPubText qui eux seront
compilé dans le processus d'externalisation.

Un des objectifs, plus secondaire, est la possibilité d'intégrer ce mécanisme
en tache Ant, ou un plug in maven.

\subsection{Une nouvelle extension de WikittyPub}

Les wikittyPubText qui auront été externalizé, et donc compilé, doivent toujours 
être considéré comme des wikitty, avec toutes les propriétés liées. Pour ce 
faire on définit une extension de WikittyPubText qui contiendra un champ nommé 
ByteCode de type tableau de byte.

Ainsi pour ce type de wikitty d'un coté on aura le code original du 
wikittyPubText dans le champ content et le code java compilé dans le champ
ByteCode. Cela permet d'avoir l'original dans le cas où l'on voudrait écrire
un nouveau wikitty avec le même code, ou dans le cas d'une surcharge si 
multi-contexte.

\subsection{Mécanisme de transformation/compilation}

Tout le mécanisme de transformation des WikittyPubText en WikittyPubTextCompiled
doit doit respecter la finalité de ces wikitty, c'est à dire que le code qu'ils
contiennent soit exécutable et exécuté. 

Quelque soit le contenu des WikittyPubText ils sont compilé, normalement pour les
langages visés, tel le javascript on ne peut pas le compiler comme on le pourrait 
pour du java. Ce qui est fait c'est que ce code est encapsulé dans un morceau
de java qui invoque le script engine chargé de son exécution. 

Pour le java comme pour l'évaluation on accepte uniquement un corps de méthode,
corps encapsulé dans la même méthode que pour du code non java:


\begin{lstlisting}
package org.nuiton.wikitty.publication;

import java.util.Map;

public abstract class AbstractDecoredClass {
    public abstract Object eval(Map<String, Object> bindings) throws Exception ;
}
\end{lstlisting}

Donc dans les deux cas le code java qui décore est presque le même, dans le 
cas de java on dépile les éléments de binding avant de remplir le corps de la 
méthode, et dans le cas de wikittyPubText on délègue les bindings à l'évaluator.
Des exemples de code décoré en annexe.

Et donc dans le processus d'évaluation par Wikitty Publication de ces 
WikittyPubTextCompiled on invoquera directement la méthode eval en lui passant
les bindings.

En dehors de la compilation et décoration des wikitty, le mécanisme est 
relativement simple. On sélectionne tous les wikitty du dossier local où on
se trouve avec le service file system, on créer un dossier temporaire, 
on y écrit les wikitty, on compile ceux qui doivent l'être, on y met le fichier
wikitty original avec le code qu'il contient, un .java et le .class, puis le 
dossier est packager en jar, et le dossier temporaire supprimé.


\subsection{Sauvegarde des propriétés des wikitty}

De la même façon que pour le stockage des wikitty sur file system afin de 
pouvoir réutiliser les wikitty et garantir que ce sont toujours des wikitty,
il faut une inversibilité de la sauvegarde, et donc conserver des propriétés des
wikitty.

Contrairement à la sauvegarde des wikitty sur File System, ici on peut se 
permettre de centraliser toutes les informations dans les mêmes fichiers de 
propriétés. 

Les propriétés conservées seront: les id, les versions ainsi que les MimeType
des fichiers correspondant. On utilise deux fichiers de propriétés distinct, 
un pour les id et les localisations des wikitty, et un autre pour les autres 
propriétés.

Exemple: % plutot mettre une image

jar:
\begin{verbatim}
+jar
|id.properties
|meta.properties
|+wp
||Wiki.wp
||Wiki.java
||Wiki.class
||Logo.png
||Stat.wp
||Stat.java
||Stat.class
||Home.wp
||Home.java
||Home.class
||WikiMenu.wp
||WikiMenu.java
||WikiMenu.class
||+dossier
|||dummy.wp
|||dummy.java
|||dummy.class
|||uneAutreImage.jpg
|||uneImage.jpg
\end{verbatim}

Fichiers de propriétés correspondant:


ids.properties:
\begin{verbatim}
62c7d0cd-91c2-4607-bdcf-f0cac6c78d6e=wp/Wiki
4120d1ba-18ac-4ac5-a329-7f515f704fd0=wp/Logo
8225d3cb-beb3-4a85-a4ea-4b0e660fab96=wp/Stat
1968c510-6ba6-4de7-bbb0-67e029119520=wp/WikiMenu
2ca85aa8-2869-4ad6-8976-c0b0b5526934=wp/dossier/dummy
12f55beb-af63-49b1-9d74-62131bd0a67a=wp/Home
b7599025-f425-4dcf-88e6-55e154430d7e=wp/dossier/uneAutreImage
e443dbbe-b461-41bd-b5b6-b612e964cb0d=wp/dossier/uneImage
\end{verbatim}


meta.properties
\begin{verbatim}
#
#Tue Jul 12 11:19:46 CEST 2011
8225d3cb-beb3-4a85-a4ea-4b0e660fab96.mime=text/javascript
4120d1ba-18ac-4ac5-a329-7f515f704fd0.mime=image/png
12f55beb-af63-49b1-9d74-62131bd0a67a.mime=text/javascript
1968c510-6ba6-4de7-bbb0-67e029119520.mime=text/javascript
62c7d0cd-91c2-4607-bdcf-f0cac6c78d6e.mime=text/javascript
e443dbbe-b461-41bd-b5b6-b612e964cb0d.mime=image/jpg
b7599025-f425-4dcf-88e6-55e154430d7e.mime=image/jpeg
2ca85aa8-2869-4ad6-8976-c0b0b5526934.mime=text/javascript
12f55beb-af63-49b1-9d74-62131bd0a67a.version=1.0
1968c510-6ba6-4de7-bbb0-67e029119520.version=1.0
2ca85aa8-2869-4ad6-8976-c0b0b5526934.version=1.0
4120d1ba-18ac-4ac5-a329-7f515f704fd0.version=1.0
62c7d0cd-91c2-4607-bdcf-f0cac6c78d6e.version=1.0
b7599025-f425-4dcf-88e6-55e154430d7e.version=1.0
e443dbbe-b461-41bd-b5b6-b612e964cb0d.version=1.0
8225d3cb-beb3-4a85-a4ea-4b0e660fab96.version=1.0
\end{verbatim}


Sous l'id on sauvegarde le chemin pour retrouver le wikitty correspondant, 
avec le mimeType qui permet de retrouver l'extension du fichier, qui sera sauvegardé
aussi dans le jar, on aura ainsi le fichier original un ".js" par exemple, 
et à coté les ".java" et ".class" correspondant aux versions décorées, compilées	.


\subsection{Wikitty Service sur jar}

Une fois que les wikitty sont sauvegardés dans un jar, on doit pouvoir les 
récupérer pour les utiliser comme n'importe quels autres wikitty. Pour celà il 
faut donc un wikitty service qui s'appuiera sur un jar et qui exposera les 
wikitty de la même façon.

Ce service propose donc les méthodes de recherche, et de restauration de wikitty,
mais évidement pas les méthodes de sauvegarde, puisque le but est bien d'avoir
des wikitty non modifiable.

La création de ce service à été facilité par l'existence du service sur file
system, puisque le fonctionnement est relativement le même, pour la recherche
on ne délègue pas à une base de donnée, on manipule directement les objets en
java, donc pour la recherche c'est le même mécanisme que pour le service sur
file system.

Ce service sur jar peut s'appuyer sur plusieurs jar, dans sa configuration 
il prend un dossier, dossier qui contient des jar, et il construit un index 
des wikitty disponible avec les fichiers de propriétés. 


