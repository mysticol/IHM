\section{Annexes}

\subsection*{Bindings}

Voici les éléments de bindings disponible dans le moteur d'évaluation:
	
\begin{itemize}
\item wpEval, correspond à l'instance de la classe java qui évalue courament le wikittyPub
\item wpSubContext, correspond à l'instance de collection contenant 
\item wpPage, nom de la page 
\item wpWikitty, wikitty en cours d'évaluation
\item wpContext, correspond à l'instance de la classe java du context (voir interface)
\end{itemize}

Interface du context:
\lstset{ %
language=JAVA,                % the language of the code
basicstyle=\footnotesize,       % the size of the fonts that are used for the code               % where to put the line-numbers
  % the size of the fonts that are used for the line-numbers
                  % the step between two line-numbers. If it's 1, each line 
keywordstyle=\color[rgb]{1,0,1},
                      % will be numbered
numbersep=5pt,                  % how far the line-numbers are from the code
showspaces=false,               % show spaces adding particular underscores
showstringspaces=false,         % underline spaces within strings
showtabs=false,                 % show tabs within strings adding particular underscores
tabsize=2,                      % sets default tabsize to 2 spaces
captionpos=b,                   % sets the caption-position to bottom
breaklines=true,                % sets automatic line breaking
breakatwhitespace=false,        % sets if automatic breaks should only happen at whitespace
title=\lstname,                 % show the filename of files included with \lstinputlisting;
                                % also try caption instead of title
escapeinside={\%*}{*)},         % if you want to add a comment with
morekeywords={project,modelVersion,groupId,description,build, plugin,plugins,configuration, applicationName, wikittyServiceUrl, artifactId,serverID,uploadUrl} 
}


\begin{lstlisting}
package org.nuiton.wikitty.publication;

import org.nuiton.wikitty.WikittyProxy;
import org.nuiton.wikitty.WikittyService;

import javax.servlet.http.HttpServletRequest;
import javax.servlet.http.HttpServletResponse;
import java.util.List;
import java.util.Map;

public interface PublicationContext {

    HttpServletRequest getRequest();
    HttpServletResponse getResponse();
    WikittyProxy getWikittyProxy();
    String makeUrl(String url);
    WikittyService getWikittyService();
    List<String> getMandatoryArguments();
    String getArgument(String name, String defaultValue);
    String getContentType();
    void setContentType(String contentType);
    String toString();
    Map<String,String> getArguments();
}

\end{lstlisting}

\subsection*{struts.xml}

Fichier de configuration xml de struts (partiel):

\lstset{ %
language=XML,                % the language of the code
basicstyle=\footnotesize,       % the size of the fonts that are used for the code               % where to put the line-numbers
  % the size of the fonts that are used for the line-numbers
                  % the step between two line-numbers. If it's 1, each line 
keywordstyle=\color[rgb]{0,0,1},
                      % will be numbered
numbersep=5pt,                  % how far the line-numbers are from the code
showspaces=false,               % show spaces adding particular underscores
showstringspaces=false,         % underline spaces within strings
showtabs=false,                 % show tabs within strings adding particular underscores
tabsize=2,                      % sets default tabsize to 2 spaces
captionpos=b,                   % sets the caption-position to bottom
breaklines=true,                % sets automatic line breaking
breakatwhitespace=false,        % sets if automatic breaks should only happen at whitespace
title=\lstname,                 % show the filename of files included with \lstinputlisting;
                                % also try caption instead of title
escapeinside={\%*}{*)},         % if you want to add a comment with
morekeywords={project,modelVersion,groupId,description,build, plugin,plugins,configuration, applicationName, wikittyServiceUrl, artifactId,serverID,uploadUrl} 
}



\begin{lstlisting}
<?xml version="1.0" encoding="UTF-8"?>
<!DOCTYPE struts PUBLIC 
	  "-//Apache Software Foundation//DTD Struts Configuration 2.0//EN"
	  "http://struts.apache.org/dtds/struts-2.0.dtd">
<struts>
    <constant name="struts.ognl.allowStaticMethodAccess" value="true" />
    <constant name="struts.enable.SlashesInActionNames" value="true" />

    <!-- Define a package for the restricted area must be logged to access -->
    <package name="restrictedArea" extends="publicArea">
        <interceptors>
            <interceptor name="login"
                class="org.nuiton.wikitty.publication.ui.interceptor.LoginInterceptor">
                <param name="error">/login_input.action</param>
            </interceptor>
            <interceptor-stack name="restrictedAreaStack">
                <interceptor-ref name="login" />
                <interceptor-ref name="publicAreaStack" />
            </interceptor-stack>
        </interceptors>
        <default-interceptor-ref name="restrictedAreaStack" />
    </package>
    <!-- Action aviable only to logged user extends="restrictedArea" -->
    <package name="publication" extends="publicArea">
        <action name="*/*/raw/*"
            class="org.nuiton.wikitty.publication.ui.action.PublicationActionRaw">
            <param name="contextData">{1}</param>
            <param name="contextApps">{2}</param>
            <param name="args">{3}</param>
            <result type="stream">
                <param name="contentType">${mimeType}</param>
                <param name="inputName">inputStream</param>
            </result>
        </action>

        <action name="*/*/eval/*"
            class="org.nuiton.wikitty.publication.ui.action.PublicationActionEval">
            <param name="contextData">{1}</param>
            <param name="contextApps">{2}</param>
            <param name="args">{3}</param>
            <result type="stream">
                <param name="contentType">${contentType}</param>
                <param name="inputName">inputStream</param>
            </result>
        </action>
    </package>
</struts>
\end{lstlisting}



\subsection*{Décoration de code pour compilation}


\subsubsection*{Javascript}

\lstset{ %
language=JAVA,                % the language of the code
basicstyle=\footnotesize,       % the size of the fonts that are used for the code               % where to put the line-numbers
  % the size of the fonts that are used for the line-numbers
                  % the step between two line-numbers. If it's 1, each line 
keywordstyle=\color[rgb]{1,0,1},
                      % will be numbered
numbersep=5pt,                  % how far the line-numbers are from the code
showspaces=false,               % show spaces adding particular underscores
showstringspaces=false,         % underline spaces within strings
showtabs=false,                 % show tabs within strings adding particular underscores
tabsize=2,                      % sets default tabsize to 2 spaces
captionpos=b,                   % sets the caption-position to bottom
breaklines=true,                % sets automatic line breaking
breakatwhitespace=false,        % sets if automatic breaks should only happen at whitespace
title=\lstname,                 % show the filename of files included with \lstinputlisting;
                                % also try caption instead of title
escapeinside={\%*}{*)},         % if you want to add a comment with
morekeywords={project,modelVersion,groupId,description,build, plugin,plugins,configuration, applicationName, wikittyServiceUrl, artifactId,serverID,uploadUrl} 
}


Ce code contenu dans un WikittyPubText avec le type mime text/javascript:

\begin{lstlisting}
try {

var result =
"    <a href='"+wpContext.makeUrl("/eval/Wiki/Home")+"'>Home</a>\n"+
"    <a href='"+wpContext.makeUrl("/eval/Wiki/Stat")+"'>Stat</a>\n";

wpContext.setContentType("text/html");
result;
} catch (eee) {
wpContext.setContentType("text/plain");
eee
}
\end{lstlisting}


Une fois décoré et inclus dans une classe java devient: 

\begin{lstlisting}
import org.apache.commons.logging.Log;
import org.apache.commons.logging.LogFactory;
import org.nuiton.wikitty.ScriptEvaluator;
import org.nuiton.wikitty.publication.AbstractDecoredClass;
import org.nuiton.wikitty.entities.*;
import org.nuiton.wikitty.publication.entities.*;
import org.nuiton.wikitty.publication.*;
import java.util.*;

public class WikiMenu extends AbstractDecoredClass {
    public Object eval(Map<String, Object> bindings) throws Exception {
        Object result = null;
        String content = "try {\n\nvar result =\n\"    <a href='\"+wpContext.makeUrl(\"/eval/Wiki/Home\")+\"'>Home</a>\\n\"+\n\"    <a href='\"+wpContext.makeUrl(\"/eval/Wiki/Stat\")+\"'>Stat</a>\\n\";\n\nwpContext.setContentType(\"text/html\");\nresult;\n} catch (eee) {\nwpContext.setContentType(\"text/plain\");\neee\n}\n";
        String mimeType = "text/javascript";
        String criteriaName = "elt_id:b48bb6c6-b38d-4abe-b8d4-97ad29457fa1";
        result = ScriptEvaluator.eval(null, criteriaName, content, mimeType,
                bindings);
        return result;
    }
}
\end{lstlisting}


\subsubsection*{Java}

Et pour un code java tel que :

\begin{lstlisting}
return "Hello world on "+wpPage;
\end{lstlisting}

Une fois décoré pour la compilation devient :

\begin{lstlisting}
import org.apache.commons.logging.Log;
import org.apache.commons.logging.LogFactory;
import org.nuiton.wikitty.ScriptEvaluator;
import org.nuiton.wikitty.publication.AbstractDecoredClass;
import org.nuiton.wikitty.entities.*;
import org.nuiton.wikitty.publication.entities.*;
import org.nuiton.wikitty.publication.*;
import java.util.*;

public class hello extends AbstractDecoredClass {
    public Object eval(Map<String, Object> bindings) throws Exception {
        PublicationContext wpContext = (PublicationContext) bindings
                .get("wpContext");
        EvalInterface wpEval = (EvalInterface) bindings.get("wpEval");
        String wpPage = (String) bindings.get("wpPage");
        List<String> wpSubContext = (List<String>) bindings.get("wpSubContext");
        Wikitty wpWikitty = (Wikitty) bindings.get("wpWikitty");
        return "Hello world on " + wpPage;

    }
}
\end{lstlisting}

\subsubsection*{Interface graphique}

Un joli WikittyPubText le mimeType "text/html.javascript" avec ce contenu:

\begin{lstlisting}
<html>
<body>

<h1> Bonjour, ceci est une page de test qui contient du html et des balises comme dans une jsp</h1>

<p>.Le but est de pouvoir mettre du html facilement avec du code aussi.
</p>

<p>ici on va afficher le nom de la page qui est dans l'adresse et dans le binding <%=wpPage%>.</p>

Op le logo:
<img src='<%=wpContext.makeUrl("/raw/Logo")%>'/>

<%

wp_result += "<p>Ici c'est du code js</p>";

if (0<1){
wp_result += "on est rentre dans un if";
}
%>
</body>
</html>
\end{lstlisting}

Après transformation il deviendra de mimeType "text/javascript" avec ce contenu:

\begin{lstlisting}
wpContext.setContentType("text/html");
var wp_result="<html>\n";
wp_result+="<head>\n";
wp_result+="</head>\n";
wp_result+="<body>\n";
wp_result+="<h1> Bonjour, ceci est une page de test qui contient du html et des balises comme dans une jsp</h1>\n";
wp_result+="<p>Le but est de pouvoir mettre du html facilement avec du code aussi.\n";
wp_result+="</p>\n";
wp_result+="<p>ici on va afficher le nom de la page qui est dans l'adresse et dans le binding "+wpPage+".</p>\n";
wp_result+="Op le logo:\n";
wp_result+="<img src='"+wpContext.makeUrl("/raw/Logo")+"'/>\n";
result += "<p>Ici c'est du code js</p>";
if (0<1){
result+= "on est rentre dans un if";
}
wp_result+="</body>\n";
wp_result+="</html>\n";
\end{lstlisting}


