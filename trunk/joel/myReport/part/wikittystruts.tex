\subsection{Une taglib struts}

L'édition de wikitty est un élément récurent dans les applications developpée 
chez code lutin, puisque en effet Wikitty est largement utilisé comme support
de base de donnée.

Ecrire des formulaires d'édition de wikitty dans les pages jsp est devenu
quelque chose récurrent avec les actions struts lié pour l'édition des dit wikitty.

L'objectif de ce module wikitty-struts à donc été de créer une taglib permettant
la génération des formulaires plus aisée pour les wikitty. 

La tag lib devait s'appuyer sur le support de tag struts qui permet de découpler
la logique et l'affichage avec un système de template. 

La création de tag se passe donc en plusieurs étape:
\begin{itemize}
\item -Définition du tag dans un fichier tld, un descripteur xml
\item -une première classe java qui étends une classe
\item -une seconde classe java qui étends une pile d'héritage qui hérite de component
\item -l'écriture du template
\end{itemize}

expliquer comment ça marche avec un schéma des architectures tout ça

Présenter ici les différents tag rapidement avec les plus pertinent de façon plus 
détaillés.


En plus de fournir des tags pour la création de formulaire, la tag lib propose 
une action qui permet de prendre en compte les modifications de wikitty envoyé
par le formulaire. Il s'agit d'une action abstraite struts que l'utilisateur à 
besoin d'étendre pour implémenter la méthode get proxy


