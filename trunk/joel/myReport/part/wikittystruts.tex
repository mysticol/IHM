\section{Wikitty Struts}

Cette partie sur wikitty n'était pas prévu au départ, mais à l'utilisation
il s'est avéré que le dévellopement d'un tel module était nécessaire, et que
le travail précédement effectué sur la partie site de wikitty publication
pouvait rendre la tache plus rapide.


\subsection{Objectfifs}

L'édition de wikitty ou tout simplement l'utilisation de wikitty au sein de 
formulaire sont des éléments récurents dans les applications developpée 
chez code lutin, puisque en effet Wikitty est largement utilisé comme support
de base de donnée. 

L'objectif de ce module wikitty-struts à donc été de créer une taglib permettant
la génération des formulaires plus aisée pour les wikitty, afin de ne pas 
avoir à refaire ce que l'on a déjà fait pour une autre application.

Pour avoir une tag lib la plus complète possible il a été décidé d'en faire une
qui marcherait de la même façon que la taglib struts de base, et se reservirait
de ses méchanismes interne, avec les templates.

\subsection{La création d'un tag}

Pour se faire la création d'un tag se passe en plusieurs étapes :

\begin{itemize}
\item -Définition du tag dans un fichier tld, un descripteur xml
\item -écriture d'une classe pour le tag, qui étend une classe abstraite de struts
\item -écriture d'une classe pour le tag qui est construite par la précédente 
classe à l'éxécution, classe qui généralement contient de la logique
\item -l'écriture du template qui lui va construire réelement le morceau de html 
correspondant, en cherchant les éléments du tag.
\end{itemize}

% coller ici schéma 



\subsection{La taglib wikitty struts "ws"}

Les tags ainsi developpé ont deux utilisations possible, la création d'un 
formulaire d'édition de wikitty, ou la création de formulaire utilisant les 
champs de wikitty comme champs. 

La différenciation de l'utilisation des tags passent par l'utilisation du tag 
de la taglib ws:form, qui implique que dans ce cas on se trouve dans l'édition d'un
wikitty.

A l'utilisation si on met seulement le tag ws:form et la source de donné 
(le wikitty) un formulaire basic va être créé en fonction du type des champs
du wikitty, ensuite en utilisant les autres tag ou différent attribut du tag 
ws:form il est possible de choisir d'exclure de l'édition des extensions,
des champs ou de choisir le type d'affichage pour un champ donné (et identifié
par son nom complet soit nomExtention.nomChamp).

En plus de fournir des tags pour la création de formulaire, la tag lib propose 
une action qui permet de prendre en compte les modifications de wikitty envoyé
par le formulaire. Il s'agit d'une action abstraite struts que l'utilisateur à 
besoin d'étendre pour implémenter la méthode accédant au proxy.

%utilisation des tags ? avec des exemples ou hje met tout en annexe  ?

Présenter ici les différents tag rapidement avec les plus pertinent de façon plus 
détaillés.





