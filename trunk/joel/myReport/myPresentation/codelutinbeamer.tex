% This text is proprietary.
% It's a part of presentation made by myself.
% It may not used commercial.
% The noncommercial use such as private and study is free
% Dec 2007
% Author: Sascha Frank 
% University Freiburg 
% www.informatik.uni-freiburg.de/~frank/
%
% 

\documentclass[12pt,a4paper,utf8x]{beamer}
\setbeamertemplate{navigation symbols}{}

\usepackage [frenchb]{babel}
\usepackage{color}
% Pour pouvoir utiliser 
\usepackage{ucs}
\usepackage[utf8x]{inputenc}

\usepackage{amsmath}

\usetheme{Marburg}

\addtobeamertemplate{footline}{\hfill\insertframenumber/\inserttotalframenumber} 

\beamersetuncovermixins{\opaqueness<1>{25}}{\opaqueness<2->{15}}
\begin{document}
\title{Wikitty Publication}  
\author{Manoël Fortun}

\begin{frame}
\titlepage
\end{frame}

\begin{frame}\frametitle{Sommaire}\tableofcontents
\end{frame} 


\section{Wikitty} 
\subsection{C'est quoi ?}
\begin{frame}\frametitle{C'est quoi ?} 
\begin{itemize}
\item Base de données orientées document.
\item Clé $\to$ Valeur
\item Entités générées grace à un modèle UML
\item champ nommé: WikittyPubText.name
\end{itemize}
\end{frame}

\subsection{Wikitty Service}
\begin{frame} \frametitle{Wikitty Service} 
\begin{itemize}
\item Restauration 
\item Sauvegarde
\item Recherche
\item Configuré par fichier de propriétés
\item Cache la réalité de la base de données
\end{itemize}
\end{frame}


\section{Scripting - JSR 223}
\subsection{Concept}
\begin{frame}\frametitle{Concept}
Interprétation de langage dans un autre langage.
Le javascript dans du java
\end{frame}
\subsection{Bindings}
\begin{frame}\frametitle{Bindings}
Possibilité des bindings
appel au méthode java
invocation
?
exemple
\end{frame}


\section{Wikitty Publication} 
\subsection{Concept}
\begin{frame}\frametitle{Concept}

Implémentation de la jsr
s'appuie sur wikitty
prototype
On y stocke une application 
que l'on peut modifier comme un wiki
application qui exploite un wikitty
donc peut se modifier
SKYNET !
\end{frame}




\section{Wikitty Struts} 

\begin{frame}\frametitle{Concept}
Une tag lib pour
la création de formulaire d'édition de wikitty
ou formulaire utilisant les valeurs d'un wikitty

\end{frame}

\subsection{Fonctionnalités}
\subsubsection{Synchronisation entre service}
\begin{frame}\frametitle{Synchronisation entre service}
comment ça marche 
commen sync
\end{frame}

\subsubsection{Externalisation}
\begin{frame}\frametitle{Externalisation}
dans un jar tout ça
à partir d'un wikitty service
\end{frame}


\subsubsection{Moteur d'évaluation}
\begin{frame}\frametitle{Moteur d'évaluation}
Les mimes type pour les interfaces
Pour les types à évaluer etc 
COntext data
COntex Apps
format des urls
\end{frame}

\subsubsection{Migration vers struts}
\begin{frame}\frametitle{Migration vers struts}
\begin{itemize}
\item mechanisme de login logout
\item action struts/interceptor
\item suport struts pour les sessions et tout
\end{itemize}
\end{frame}


\subsection{Nouveau type de wikitty service}
\begin{frame}\frametitle{Nouveau type de wikitty service}
\begin{itemize}
\item wikitty service sur système de fichier
\item wikitty service sur jar
\item wikitty service fallback
\end{itemize}
%insert image

\end{frame}

\subsection{Maven}
\begin{frame}\frametitle{Maven}
\begin{itemize}
\item Facilité l'utilisation
\item pas besoin de chercher dans les classes java
\item immediat
\end{itemize}
%pom exemple
%liste des goals

\end{frame}	


\end{document}
