% This text is proprietary.
% It's a part of presentation made by myself.
% It may not used commercial.
% The noncommercial use such as private and study is free
% Dec 2007
% Author: Sascha Frank 
% University Freiburg 
% www.informatik.uni-freiburg.de/~frank/
%
% 

\documentclass[12pt,a4paper,utf8x]{beamer}
\setbeamertemplate{navigation symbols}{}

\usepackage [frenchb]{babel}
\usepackage{color}
% Pour pouvoir utiliser 
\usepackage{ucs}
\usepackage[utf8x]{inputenc}

\usepackage{amsmath}

\usetheme{Marburg}

\addtobeamertemplate{footline}{\hfill\insertframenumber/\inserttotalframenumber} 

\beamersetuncovermixins{\opaqueness<1>{25}}{\opaqueness<2->{15}}
\begin{document}
\title{Wikitty Publication}  
\author{Manoël Fortun}

\begin{frame}
\titlepage
\end{frame}

\begin{frame}\frametitle{Sommaire}\tableofcontents
\end{frame} 


\section{Wikitty} 
\subsection{C'est quoi ?}
\begin{frame}\frametitle{Concept} 
\begin{itemize}
\item Base de données orientées document.
\item Clé $\to$ Valeur
\item Entité généré grace à un modèle uml
\end{itemize}
\end{frame}

\subsection{Wikitty Service}
\begin{frame} \frametitle{Wikitty Service} 
Toute les interraction avec le véritable stockage masqué par le service
et souvent encapsuler dans un proxy pour en plus 

\end{frame}


\section{Scripting - JSR 223} 
\begin{frame}\frametitle{Concept}
Interprétation de langage dans un autre langage.




\end{frame}

\begin{frame}\frametitle{Bindings}
Possibilité des bindings
appel au méthode java
invocation
?


\end{frame}


\section{Wikitty Publication} 
\subsection{Concept}
\begin{frame}\frametitle{Concept}

Implémentation de la jsr
s'appuie sur wikitty
prototype

\end{frame}




\section{Wikitty Struts} 

\begin{frame}\frametitle{Concept}


\end{frame}

\end{document}
