\documentclass[12pt,a4paper,utf8x]{report}
\usepackage [french]{babel}

% Pour pouvoir utiliser 
\usepackage{ucs}
\usepackage[utf8x]{inputenc}

\usepackage{url} % Pour avoir de belles url
\usepackage {geometry}

% Pour mettre du code source
\usepackage {listings}
% Pour pouvoir passer en paysage
\usepackage{lscape}

% Pour insérer des images
\usepackage{graphicx}

% Pour pouvoir faire plusieurs colonnes
\usepackage {multicol}
% Pour crééer un index
\usepackage{makeidx}
\makeindex

% Pour gérer les liens interractifs et les signets Acrobat
\usepackage{hyperref}
\hypersetup{
pdftitle={TP 5 : JMS},
pdfauthor={Fortun, Caillaud, Dejean, Pottier},
pdfsubject={TP 5 : JMS},
bookmarks, % Création du signet
pdfstartview=FitH, % Page de la largeur de la fenêtre
colorlinks=true, % Liens en couleur
linkcolor=black, 	
anchorcolor=black, 	
citecolor=black, 	
filecolor=black, 	
menucolor=black,
runcolor=black,
urlcolor=black, 	
frenchlinks=black,
bookmarksnumbered=true, % Signet numéroté
pdfpagemode=UseOutlines, % Montre les bookmarks.
bookmarksopen =true,
}

% Pour afficher la bibliographie, mais pas, notlof (List of Figures) ni notlot (List of Tables)
\usepackage[notlof, notlot]{tocbibind}


% Pour les entetes de page
% \usepackage{fancyheadings}
%\pagestyle{fancy}
%\renewcommand{\sectionmark}[1]{\markboth{#1}{}} 
%\renewcommand{\subsectionmark}[1]{\markright{#1}} 

% Pour l'interligne de 1.5
\usepackage {setspace}
% Pour les marges de la page
\geometry{a4paper, top=1.5cm, bottom=2.5cm, left=1.5cm, right=1.5cm, marginparwidth=1.2cm}

\parskip=5pt %% distance entre § (paragraphe)
\sloppy %% respecter toujours la marge de droite 

% Pour les pénalités :
\interfootnotelinepenalty=150 %note de bas de page
\widowpenalty=150 %% veuves et orphelines
\clubpenalty=150 

%Pour la longueur de l'indentation des paragraphes
\setlength{\parindent}{15mm}

%%%% debut macro pour enlever le nom chapitre %%%%
\makeatletter
\def\@makechapterhead#1{%
  \vspace*{50\p@}%
  {\parindent \z@ \raggedright \normalfont
    \interlinepenalty\@M
    \ifnum \c@secnumdepth >\m@ne
        \Huge\bfseries \thechapter\quad
    \fi
    \Huge \bfseries #1\par\nobreak
    \vskip 40\p@
  }}

\def\@makeschapterhead#1{%
  \vspace*{50\p@}%
  {\parindent \z@ \raggedright
    \normalfont
    \interlinepenalty\@M
    \Huge \bfseries  #1\par\nobreak
    \vskip 40\p@
  }}
\makeatother
%%%% fin macro %%%%

%Couverture 

\title
{
	\Huge{TP 5 : JMS}
}
\author{Fortun Manouel, Anthony Caillaud, Vincent POTTIER\\Charles DEJEAN\\
	\vspace{45mm}
}

\date{	\today }
	


\begin{document}

\maketitle

%Remerciements

Je tiens à remercier :
et on met la liste des personnes que l'on remercie. Toto, tutu, titi. et on met la liste des personnes que l'on remercie. Toto, tutu, titi.et on met la liste des personnes que l'on remercie. Toto, tutu, titi.et on met la liste des personnes que l'on remercie. Toto, tutu, titi.


Et on met la liste des personnes que l'on remercie. Toto, tutu, titi.et on met la liste des personnes que l'on remercie. Toto, tutu, titi.et on met la liste des personnes que l'on remercie. Toto, tutu, titi.et on met la liste des personnes que l'on remercie. Toto, tutu, titi.et on met la liste des personnes que l'on remercie. Toto, tutu, titi.

%\clearpage

\tableofcontents
\clearpage

% Pour avoir un interligne de 1,5
\begin{onehalfspace}

\chapter{User Stories pour le composant dé}

En tant que Joueur, avec le composant de dés, je dois pouvoir lancer un nombre
de dé choisi et du type que je souhaite et je dois pouvoir voir le résultat
correct.
~

En tant que Joueur, avec le composant de dés, je dois pouvoir, après un jet de
dés, relancer les dés avec les mêmes caractéristiques.
~

En tant que Joueur, avec le composant de dés, je dois pouvoir changer le type de
dés ou le nombre

\clearpage


\subsubsection{User Stories pour la connectivité}

En tant que Joueur, avec le module Connectivité, je dois pouvoir rejoindre une
salle de chat.\\

En tant que Joueur, avec le module Connectivité, je dois pouvoir créer un salon
et devenir maître de salon.\\

En tant que Joueur, avec le module Connectivité, je dois pouvoir parler de façon
discrète (privée) à un autre joueur présent.\\

En tant que Joueur, avec le module Connectivité, je dois pouvoir choisir que
toutes mes conversations soient visibles par le maître de salon.\\

En tant que Joueur, avec le module Connectivité, je dois pouvoir choisir que
tous mes jets soient envoyés au Maître du salon.\\


En tant que Maître du salon, avec le module Connectivité, je dois pouvoir
choisir de ne pas voir les conversations ou les dés.\\

En tant que Joueur, avec le module Connectivité, je dois pouvoir ne pas être que
sur un salon.\\

En tant que Joueur, avec le module Connectivité, je dois pouvoir accéder à la
liste des gens sur le salon.\\

En tant que Joueur, avec le module Connectivité, je dois pouvoir, depuis ma
fiche, accéder rapidement au salon ou je suis connecté.\\



\subsubsection{User Stories pour le Joueur}

En tant que Joueur, dans la partie consultation de fiche, je dois pouvoir voir
les caractéristiques du personnage.\\

En tant que Joueur, dans la partie consultation de fiche, je dois pouvoir voir
les compétences du personnage.\\

En tant que Joueur, dans la partie consultation de fiche, je dois pouvoir voir
les autres informations du personnage, en fonction du jeu.\\

En tant que Joueur, dans la partie consultation de fiche, je dois pouvoir voir
les caractéristiques secondaires du personnage.\\

En tant que Joueur, dans la partie consultation de fiche, je dois pouvoir
sélectionner une caractéristique afin de pouvoir la modifier.\\

En tant que Joueur, dans la partie consultation de fiche, je dois pouvoir
sélectionner une compétence afin de pouvoir la modifier.\\

En tant que Joueur, dans la partie consultation de fiche, je dois pouvoir
sélectionner une caractéristique secondaire afin de pouvoir la modifier.\\

En tant que Joueur, dans la partie consultation de fiche, je dois pouvoir
passer d'une catégorie à une autre facilement.\\

En tant que Joueur, dans la partie consultation de fiche, je dois pouvoir
sélectionner la fiche consultée pour une autre partie.\\


% Pour finir l'interligne de 1,5
\end{onehalfspace}

%----------------------------------------
% Pour la bibliographie
%----------------------------------------
% Citer tous les ouvrages/références
% \nocite{*}
% Trier par ordre d'apparition
% \bibliographystyle{unsrt}
% Pour le style de la biblio
%\bibliographystyle{plain.bst}
% Ecrire la biblio ici
% \bibliography{biblio}

\printindex

\appendix


\end{document}
