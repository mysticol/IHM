\documentclass[11pt,a4paper]{article}
\usepackage{ifpdf}
\usepackage[utf8]{inputenc}
\usepackage[francais]{babel}
\usepackage[T1]{fontenc}
\usepackage[nottoc, notlof, notlot]{tocbibind}
\usepackage[unicode=true,pdftex,colorlinks=true,linkcolor=black,urlcolor=black,citecolor=black]{hyperref}
\usepackage{natbib}
\usepackage{graphicx}

\parindent 0.8cm
%\setlength{\parskip}{0.5em plus 0.2em minus 0.2em}

\title{Projet Androïd : RoadMap}
\author{Anthony \textsc{Caillaud} Manoël \textsc{Fortun} Vincent
\textsc{Pottier} Charles \textsc{Dejean}}
\date{\today}
\ifpdf
\pdfinfo {
/Author (Anthony Caillaud Manoël Fortun Vincent Pottier Charles Dejean)
/Title (RoadMapJDR)
/Subject (RoadMapJDR)
/Keywords ()
/CreationDate (D:20100329212218)
}
\fi


\begin{document}

\maketitle

\clearpage
\tableofcontents
\clearpage
\section{Concept haut-niveau du produit}

%Donner le nom que vous voulez!
AndroJDR permet de jouer à n'importe quel jeu de rôle et offre des
fonctionnalités telles que la gestion de fiches de personnages, le jet de
différents types de dés ou encore un chat entre joueur. Les utilisateurs visés
sont l'ensemble des personnes jouant régulièrement au jeux de rôles. L'outil
proposera une interface simple, intuitive et adaptée au téléphone mobile.

\section{La population d'utilisateurs}
\label{utilisateurs}

Les utilisateurs visés sont des utilisateurs novices. En effet, ceux-ci auront
une excellente connaissance du domaine mais pas d'applications similaires.
Cette application représentant une simplification de certains aspects du jeu,
elle s'addresse autant au joueur occasionnel qu'au joueur régulier.\\

L'Interface Homme-Machine devra donc comprendre des fonctionnalités bien
adaptées au type de joueur(Joueur ou MJ) et facile d'accès. Elle devra être
claire pour permettre à l'utilisateur de bien visualiser toutes les données(Le
joueur doit pouvoir bien identifier toutes les caractéristiques et les compétences de
son personnage facilement). Le fait qu'aucune application similaire n'existe,
la prise en main de l'application devra être facile et donc avoir une interface
simple et intuitive.
	

\section{Objectifs en termes d'utilisabilité}

Comme cité dans la section \ref{utilisateurs}, ces objectifs ont été fixés pour
des utilisateurs novices ayant une bonne connaissance du jeu de rôle et aucune
connaissance d'applications similaires.\\

Ces objectifs sont les suivants :

\begin{itemize}
  \item Faciliter l'accès à différentes fiches de personnages.
  \item Faciliter le jet de dés de différents type(dé 6,dé 10,etc)en fonction
  du jeu, des caractéristiques et des compétences du personnage joué.
  \item Faciliter la communication entre deux joueur(ou entre joueur et MJ)
  grâce à un chat.
  \item Tâches récurrentes: Affichage de fiches de personnages et jets de dés.
\end{itemize}

\clearpage

\section{Les contraintes de haut-niveau}

\subsection{Contraintes matérielles}

Cette application est réalisée pour une application pour téléphone mobile
tactile ou tablette graphique tactile. De plus, elle sera développée en Java et
sera donc portable sur différents système d'exploitation.

\subsection{Contraintes budgétaires}


\section{Les fonctionnalités}
\subsection{Pour tous les utilisateurs}
\subsection{Pour les joueurs}
\subsection{Pour le MJ}

\clearpage
  
\end{document}
