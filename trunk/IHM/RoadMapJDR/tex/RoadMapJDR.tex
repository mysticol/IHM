\documentclass[11pt,a4paper]{article}
\usepackage{ifpdf}
\usepackage[utf8]{inputenc}
\usepackage[francais]{babel}
\usepackage[T1]{fontenc}
\usepackage[nottoc, notlof, notlot]{tocbibind}
\usepackage[unicode=true,pdftex,colorlinks=true,linkcolor=black,urlcolor=black,citecolor=black]{hyperref}
\usepackage{natbib}
\usepackage{graphicx}

\parindent 0.8cm
%\setlength{\parskip}{0.5em plus 0.2em minus 0.2em}

\title{Projet Androïd : RoadMap}
\author{Anthony \textsc{Caillaud} Manoël \textsc{Fortun} Vincent
\textsc{Pottier} Charles \textsc{Dejean}}
\date{\today}
\ifpdf
\pdfinfo {
/Author (Anthony Caillaud Manoël Fortun Vincent Pottier Charles Dejean)
/Title (RoadMapJDR)
/Subject (RoadMapJDR)
/Keywords ()
/CreationDate (D:20100329212218)
}
\fin


\begin{document}

\maketitle

\clearpage
\tableofcontents
\clearpage
\section{Concept haut-niveau du produit}

%Donner le nom que vous voulez!
AndroJDR(nom provisoire de l'application), permet de stocker des fiches de
personnage de jeu de rôles, et à partir de ces fiches, d'effectuer des jets de
dés. L'application offre aussi des fonctionnalités de communication entre
joueurs. Les utilisateurs visés sont des rôlistes expérimentés ou novices,
disposant d'un smartsphone ou d'un tablet pc. L'outil proposera une interface
simple et intuitive, s'appuyant sur les écrans tactils adaptables pour
différents formats d'écran.

\section{La population d'utilisateurs}
\label{utilisateurs}

Les utilisateurs visés sont des utilisateurs novices. En effet, ceux-ci auront
une excellente connaissance du domaine mais pas d'applications similaires.
Cette application, représentant une simplification de certains aspects du jeu,
s'addresse donc autant à un joueur occasionnel qu'à un joueur régulier.\\

L'Interface Homme-Machine devra donc comprendre des fonctionnalités bien
adaptées aux type de joueurs(Joueur ou MJ) et faciles d'accès. Elle devra être
claire, pour permettre à l'utilisateur de bien visualiser toutes les données
: le joueur doit pouvoir identifier toutes les caractéristiques et les
compétences de son personnage facilement. Du fait qu'aucune application
similaire existe, la prise en main de l'application devra être facile et donc avoir une interface
simple et intuitive.


\section{Objectifs en termes d'utilisabilité}

Comme cités dans la section \ref{utilisateurs}, ces objectifs ont été fixés pour
des utilisateurs novices ayant une bonne connaissance du jeu de rôle et aucune
connaissance d'applications similaires.\\

Ces objectifs sont les suivants :

\begin{itemize}
  \item Faciliter l'accès à différentes fiches de personnages.
	\item Permettre la gestion de différents jeux de roles via des modèles de fiche
	et de règles
\item Permettre à des utilisateurs avancés de définir de nouveaux modèles de
fiches et de règles
  \item Faciliter les jets de dés de différents types (dé 6,dé 10,etc), en
  fonction du jeu, des caractéristiques et des compétences du personnage joué.
  \item Faciliter la communication entre deux joueurs (ou entre joueur et MJ)
  grâce à un chat.
  \item Tâches récurrentes: Affichage de fiches de personnages et jets de dés.
\end{itemize}

\clearpage

\section{Les contraintes de haut-niveau}
% Section à développer
\subsection{Contraintes matérielles}

Cette application est réalisée pour être une application pour téléphone mobile
tactile ou tablette graphique tactile. De plus, elle sera développée en Java et
sera donc portable sur différents systèmes d'exploitations.

\subsection{Contraintes budgétaires}

Les services qui pourront être utilisé par l'application devront être gratuit, le budget étant nul.

\section{Les fonctionnalités}

\subsection{Pour tous les utilisateurs}
Les fonctionnalités disponibles pour tous les utilisateurs sont les suivantes :

\begin{itemize}
  \item Accéder à et visualiser toutes ses fiches de personnages.
  \item Créer, éditer et supprimer une fiche de personnage.
  \item Effectuer des jets de dés de différents types.
  \item Effectuer des jets de dés en fonction de certaines caractéristiques et
  compétences d'un personnage au cours d'une partie.
  \item Chatter avec un ou plusieurs des joueurs ou avec le MJ.
\end{itemize}

\subsection{Pour le MJ}
Le MJ possède, quant à lui, des fonctionnalités spécifiques en plus de celles
citées précédemment.

Ces fonctionnalités sont les suivantes :

\begin{itemize}
  \item Visualiser les fiches des personnages participant à une partie.
  \item Gestion du chat et visualisation de toutes les discussions.
\item Gestion de la visualisation des jets de dés effectués par les joueurs
\item Interface facilitante pour l'affichage de nombreuses fiches en même temps,
permettant au MJ la gestion de ses pnj.
\item Possibilté de la gestion des dégats reçus par les différents joueurs.
\end{itemize}

\clearpage
  
\end{document}
