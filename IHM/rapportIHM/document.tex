\documentclass[11pt,a4paper]{report}
\usepackage{ifpdf}
\usepackage[utf8]{inputenc}
\usepackage[francais]{babel}
\usepackage[T1]{fontenc}
\usepackage[nottoc, notlof, notlot]{tocbibind}
\usepackage[unicode=true,pdftex,colorlinks=true,linkcolor=black,urlcolor=black,citecolor=black]{hyperref}
\usepackage{natbib}
\usepackage{graphicx}

\parindent 0.8cm
%\setlength{\parskip}{0.5em plus 0.2em minus 0.2em}

\title{Projet Androïd : Rapport}
\author{Anthony \textsc{Caillaud} Manoël \textsc{Fortun} Vincent
\textsc{Pottier} Charles \textsc{Dejean}}
\date{\today}
\ifpdf
\pdfinfo {
/Author (Anthony Caillaud Manoël Fortun Vincent Pottier Charles Dejean)
/Title (RoadMapJDR)
/Subject (RoadMapJDR)
/Keywords ()
/CreationDate (D:20100329212218)
}
\fin


\begin{document}

\maketitle

\clearpage
\tableofcontents
\clearpage
\section{Introdution}

Ce rapport reprend la suite de la Roadmap que nous avions définis pour notre
projet d'ihm autour d'une application pour le jeu de rôle sous androïd.
Dans la continuité du précédent rapport nous avons suivis la méthode LUCID pour
la conception et réalisation de notre application. Ce rapport reprendra donc là
ou nous nous étions arrêté, c'est à dire à l'étape 2 de LUCID, et finira à 6
avec le plan d'évaluation si nous étions dans un contexte réel.
Avant chaque étape nous rappelerons les enjeux de l'épate et notre
interprétation.

\clearpage

\section{Lucid 2}

\subsection{Définition}

Cette étape de la méthode permet de compléter la première étape, et donc de
receuillir en fonction des catégories d'utilisateur et du concept de
l'application les attentes des utilisateurs. D'obtenir aussi les choses qu'ils
jugent indispensable, la façon qu'ils auraient d'appréhender l'application via
des scénario, et donc les implications en termes de fonctionnalités.

\subsubsection{Approche des utilisateurs}


Trois grandes réactions, classique:
Gens intéressé pour l'utilisation et l'enregistrement de ce qu'ils veulent:
contrainte d'enrichissement facile, pas besoin de rentrer dans du technique


Gens curieux de voir ce que ça va donner

%%Note pour la relecture c'est important de conserver le fond et une partie de
% la forme du discourt qui suit. Ca fera rire le prof, ça fait du point en
% plus. Et merde on est des gros clampins. 
irréductible accrocher à la symbolique du support physique de leur dés et de
leur fiche. Parce que lancer un téléphone sur un joueur c'est vachement plus
onéreux que lui lancer ses dés. Même remarque sur le fait de poignarder un
joueur avec téléphone plutot que le sacro saint crayon de bois servant à
l'édition de sa fiche.


Les gens veulent des dés qui s'affiche et que l'on peut customiser, les gen
veulent que l'application rêvete des thèmes différent en fonction du jeu.
Pouvoir convertir leur fiche en vrai fiche à imprimer avec des belles fiches
classique, voir pouvoir customiser la sortie en fiche papier.

Gens inquiet par le coté connectivité chat, une partie ne comprend pas le
concept tel qu'il a été écrit.


Gens préférant une telle application sur tablette pc, plus adapté d'après eux.

Faut que la synchronisation sur pc soit facile, via un outil de synch
voir même un site web de relais pour le stockage facile

Des gens on demandé si ce n'étais pas un jeu que nous faisions.

Les gens trouvent que Joël il est tout petit.

Interprétation au niveau IHM ? là nan je vois pas ce qu'on pourrait dire.

\subsubsection{Userstory}

Coller ici les users story en les ordonnants et tout le tintouin.


\section{Lucid 3}

\subsection{Définition}

Cette étape de la méthode est la première étape visuelle, c'est à se moment là
que on se sert des infos compilées des précédentes étapes pour pouvoir dessiner
quelques chose qui aura un chance de plaire aux utilisateurs finaux.
On se sert des indications utilisateurs visuel, de ce qui est possible et des
users stories pour construire les interractions entre écran, enchainement.
Pendant cette étape il est très important que sur les exquisses d'interfaces il
n'y ait pas d'ambiguité possible, tout élément doit être nommé pertinement et
correspondre à quelque chose qui à du sens. Le label ``accès fiche'' est
préférable au label ``ponay''. 


\subsubsection{Esquisses interface}

Là on va coller l'interface dessiné dans le google docs, avec quelques
commentaires quand même on est pas des animaux merde.
Plus parler du look and feel
Que pour le moment nous avons choisi d'utiliser celui de base contenu dans
android puisque sobre et efficace et que l'aspect visuel customisable pour tous
les aspect de jeu de role n'est pas une priorité. Que la première version
s'attardera surtout sur le fonctionnel et ergonomie de l'affichage des données,
de plus la propention de look and feel à changer cette ergonomie trop important
pour être gérer de façon négligeable.

\section{Lucid 4}

\subsection{Définition}

Cette étape de la méthode est une étape presque introspective, elle permet de
revenir, sur les étapes précédentes par rapport au retour client sur le
prototypage. Et par la même de pouvoir commencer à établir le plan de
construction de l'application au niveau des fonctionnalités. Il s'agit de
plannifier les taches de dévellopement. 

\subsubsection{Notre organisation}

Mentir sur les taches l'organisation en brique qui s'enboite comme des légos et
tout le tintouint
//Revoir la définition de cette étape


\clearpage

\section{Lucid 5}

\subsection{Définition}

Cette étape de la méthode est l'implémentation de l'application. C'est donc dans
cette étape que l'interface est écrite en parrallèle avec le code effectif des
fonctions. C'est aussi le moment où l'on commence à écrire les documentations et
quelques structures communautaire autour de l'outil. C'est vraiment à ce moment
là que l'on rentre dans un cycle logiciel avec du concret informatiquement
parlant.

\subsubsection{L'implémentation}

Là on va parler de l'organisation du travail, à base de séparation en couche de
l'application. Partie de l'application en brique qui s'emboite et des adapteurs
pour une réutilisation.
Ici on parle d'android et de ses défauts qu'il faut que vous me listiez. 

\subsubsection{IHm android tout ça}

\section{Lucid 6}

\subsection{Définition}

Cette étape de la méthode est l'évaluation externe du projet. Elle s'accompagne
généralement de la diffusion d'une première version public du produit qui
normalement est proche de la version finale tant par l'aspect que par ses
fonctionnalités.
Cette étape permet de lever les dernières remarques du public visé, et de
corriger les bugs induit par la différence d'utilisation des utilisateurs finaux
par rapport à l'utilisation des devellopeurs.


\subsection{Plan d'évalution}

Mise en place de trois type de niveau de communication avec la communauté,
une pour les MJ et donc utilisateur amené à faire évoluer les modèles de fiche
modèle de système, ergonomie pour l'affichage de plein de chose

Utilisateur simple utilisateur voulant importer ses fiches, synchro ses fiches
simplement, faire ses jets, niveau ergonomie des choses

Utilisateur novice découvrant l'outil donc très accompagné avec des interfaces
didactique, être a l'écoute des choses qu'ils voudraient comprenndre.

(But finale est la distiction de deux type d'utilisateur simple et avancé)

\subsubsection{Utilisateur avancé (MJ)}

\subsubsection{Utilisateur simple (joueur)}

\subsubsection{Utilisateur novice (joueur)}
  

\section{Conclusion}
Lucid ethno centré
méthode pouvant entré en conflit avec les méthodes de dévellopement classic
d'ingénièrie logiciel.
On gagne en ethno sympathie, donc plus de chance que le produit soit accepté et
conforme à ce que l'utilisateur final va vouloir, mais le temps nécessaire plus
long que un cycle ``classique'' donc potentiellement plus d'argent à investir
pour le client. L'idéal serait un mix des deux.
Un autre soucis est l'aspect communication avec le client, et donc l'analyse de
ses besoins et la transcription en élément informatique concret et utilisable,
mais cet aspect social, communication des dévellopeurs généralement il est assez
peu dévellopé, donc méthode complexe dans certain cas. Ce ne saurait pas le cas
si le marché des informaticiens n'étais pas saturé(trusté) d'associals. 
  
  
\end{document}
