\documentclass[11pt,a4paper]{report}
\usepackage{ifpdf}
\usepackage[utf8]{inputenc}
\usepackage[francais]{babel}
\usepackage[T1]{fontenc}
\usepackage[nottoc, notlof, notlot]{tocbibind}
\usepackage[unicode=true,pdftex,colorlinks=true,linkcolor=black,urlcolor=black,citecolor=black]{hyperref}
\usepackage{natbib}
\usepackage{graphicx}

\parindent 0.8cm
%\setlength{\parskip}{0.5em plus 0.2em minus 0.2em}

\title{Projet Androïd : Rapport}
\author{Anthony \textsc{Caillaud} Manoël \textsc{Fortun} Vincent
\textsc{Pottier} Charles \textsc{Dejean}}
\date{\today}
\ifpdf
\pdfinfo {
/Author (Anthony Caillaud Manoël Fortun Vincent Pottier Charles Dejean)
/Title (RoadMapJDR)
/Subject (RoadMapJDR)
/Keywords ()
/CreationDate (D:20100329212218)
}
\fin


\begin{document}

\maketitle

\clearpage
\tableofcontents
\clearpage
\section{Introdution}

Ce rapport reprend la suite de la Roadmap que nous avions définis pour notre
projet d'ihm autour d'une application pour le jeu de rôle sous androïd.
Dans la continuité du précédent rapport nous avons suivis la méthode LUCID pour
la conception et réalisation de notre application. Ce rapport reprendra donc là
ou nous nous étions arrêté, c'est à dire à l'étape 2 de LUCID, et finira à 6
avec le plan d'évaluation si nous étions dans un contexte réel.
Avant chaque étape nous rappelerons les enjeux de l'épate et notre
interprétation.

\clearpage

\section{Lucid 2}
\label{utilisateurs}



\section{Lucid 3}

\section{Lucid 4}

\subsection{Pour tous les utilisateurs}
Les fonctionnalités disponibles pour tous les utilisateurs sont les suivantes :

\begin{itemize}
  \item Accéder à et visualiser toutes ses fiches de personnages.
  \item Créer, éditer et supprimer une fiche de personnage.
  \item Effectuer des jets de dés de différents types.
  \item Effectuer des jets de dés en fonction de certaines caractéristiques et
  compétences d'un personnage au cours d'une partie.
  \item Chatter avec un ou plusieurs des joueurs ou avec le MJ.
\end{itemize}

\subsection{Pour le MJ}
Le MJ possède, quant à lui, des fonctionnalités spécifiques en plus de celles
citées précédemment.

Ces fonctionnalités sont les suivantes :

\begin{itemize}
  \item Visualiser les fiches des personnages participant à une partie.
  \item Gestion du chat et visualisation de toutes les discussions.
\item Gestion de la visualisation des jets de dés effectués par les joueurs
\item Interface facilitante pour l'affichage de nombreuses fiches en même temps,
permettant au MJ la gestion de ses pnj.
\item Possibilté de la gestion des dégats reçus par les différents joueurs.
\end{itemize}

\clearpage

\section{lucid 5}




\section{Lucid 6}

\subsection{Définition}

Cette étape de la méthode est l'évaluation externe du projet. Elle s'accompagne
généralement de la diffusion d'une première version public du produit qui
normalement est proche de la version finale tant par l'aspect que par ses
fonctionnalités.
Cette étape permet de lever les dernières remarques du public visé, et de
corriger les bugs induit par la différence d'utilisation des utilisateurs finaux
par rapport à l'utilisation des devellopeurs.


\subsection{Plan d'évalution}

Mise en place de trois type de niveau de communication avec la communauté,
une pour les MJ et donc utilisateur amené à faire évoluer les modèles de fiche
modèle de système, ergonomie pour l'affichage de plein de chose

Utilisateur simple utilisateur voulant importer ses fiches, synchro ses fiches
simplement, faire ses jets, niveau ergonomie des choses

Utilisateur novice découvrant l'outil donc très accompagné avec des interfaces
didactique, être a l'écoute des choses qu'ils voudraient comprenndre.

(But finale est la distiction de deux type d'utilisateur simple et avancé)

\subsubsection{Utilisateur avancé (MJ)}

\subsubsection{Utilisateur simple (joueur)}

\subsubsection{Utilisateur novice (joueur)}
  
\end{document}
